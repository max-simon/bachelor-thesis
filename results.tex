\chapter{Results}
In the first part of the project, we investigated the cause of the falling of FAK. Therefore, we want to begin the presentation of our results in \autoref{results:umbrella} with the main finding of these studies, namely that the falling is not due to an underestimation of the binding energy in \martini{}. Afterwards, we analyse the observations of FAK in solution to develop a reference state for later comparisons (\autoref{sec:fak_sol}). Before we consider FAK molecules on membranes, the applied stabilising force and its impact on used observables is examined in \autoref{forceana}. The investigation of FAK molecules on the membrane is split up into two parts. In \autoref{membrane:chapter}, the obtained conformational changes of FAK molecules bound to a \pip{}-containing membrane are presented. In the last section of this chapter, \autoref{multiProt}, the focus is then on interactions between multiple FAK molecules.
%
%
\section{Free energy profile of basic patch}
From the umbrella simulations the free energy profile was retrieved using \gromacs{} WHAM. In \autoref{bild} the average of the five copies together with the standard deviation can be found for each forcefield. The range between 6$\si{\nano\metre}$ and 7$\si{\nano\metre}$ was set to 0.\\
As you can see in the plot, the results for C36 and Martini are very similar: both show a depth of $\approx 17.5 \si{\kilo\cal/\mole}$ and the same slope between 3$\si{\nano\metre}$ and 4$\si{\nano\metre}$. However it seems like the Martini profile is shifted to smaller z-distances. One reason for this is, that due to the coarse graining the COM distances alters (in our case the COM distance of Protein and \pip{} decreased by 0.1$\si{\nano\metre}$ after coarse graining). Beyond 4$\si{\nano\metre}$ there is a larger difference: while the free energy profile flattens out in C36, a kink can be found in Martini. This is explained by the different treatment of long range electrostatic interactions. For this Martini just uses a cut off radius, C36 uses PME.\\
Also Martini with PW uses PME for long range electrostatic interactions and indeed this fits much better to the profile observed in C36. However the binding strength is drastically underestimated when using PW. The same effect was observed by simulations of a single FAK on a \pip{} containing membrane. Here the kinase dissociated from the membrane, which is probably due to an underestimation of the \pip{} binding site in the kinase.\\
The results indicate, that Martini is a suitable simplification for simulating FAK on membranes, which allow much longer simulations of much larger systems.
\clearpage
%
%
\section{FAK in solution}
% TODO: link to methods
In this section the conformation of FAK in absence of \pip{} and other FAK molecules is analyzed. For this purpose the simulation data of setup 1 were used (see \autoref{setup:setup1}).\\
\\
First the COM distances of F1 to the N-lobe ($d_\text{F1-N}$) and F2 to the C-lobe ($d_\text{F2-C}$) are considered. In \autoref{tobeadded} a hexagonal binning plot of both values can be found, which indicates, that there are two different states: one in which $d_\text{F2-C}$ is larger (spot 1) and $d_\text{F1-N}$ smaller and one the other way around (spot 2). The systems starts in spot 1 and goes to spot 2 after $7\,\si{\micro\second}$ (this transition goes along with several frequent transitions), where it stays until the end of the simulation.\\
\textcite{jing} performed equivalent simulations in C36 and obtained
\begin{align}
	d_\text{F1-N} &= (3.30 \pm 0.40\,\text{std})\si{\nano\metre}\\
	d_\text{F2-C} &= (3.15 \pm 0.15\,\text{std})\si{\nano\metre}
\end{align}
These values can not be classified in one of the two spots as both distances are lower than those obtained in Martini.\\
\\
A contact map of the interface between the FERM domain and the kinase for frames of spot 2 can be found in \autoref{tobeadd}. Two contact areas can be identified. The first one (area 1) is located between F1 and the N-lobe/activation loop. It shows i.e. contacts between \acid{Y}{576} and \acid{Y}{577} and residues of the FERM domain. The minimal distance in this area, between residue \acid{H}{41} and \acid{Y}{576}, is $0.45\,\si{\nano\metre}$ with an RMSF value of $0.03\,\si{\nano\metre}$. This reflects the burying of the activity regulating residues in closed state.\\
The second contact area (area 2) is located between F2 and the C-lobe. The spots occur around the residues \acid{Y}{180} and \acid{D}{200} of F2 as well as \acid{F}{596} and \acid{R}{665}. The minimal distance in this area occurs between \acid{Y}{180} and \acid{F}{596} with $0.45\,\si{\nano\metre}$ and an RMSF value of $0.02\,\si{\nano\metre}$. These two residues have been reported as important actors in the interface by showing, that a mutation disturbs the interface and enhances the activation of FAK.\\
The linker show contacts with both domains. Interestingly the minimal distance in the marked areas (area 3 and area 4) occur between the autophosphorylation site \acid{Y}{397} and \acid{H}{58} ($0.45\,\si{\nano\metre}$, RMSF $0.03\,\si{\nano\metre}$) in F1 or \acid{Y}{576} ($0.50\,\si{\nano\metre}$, RMSF $0.10\,\si{\nano\metre}$) in the kinase. This is consistent with the concept, that autophosphorylation is prevented in closed conformation by a binding of the linker to the FERM domain.\\
\\
In contrast to \autoref{tobeadd}, the contact map for frames of spot 1 show less contacts between F2 and the C-lobe, i.e. around the mentioned residues \acid{Y}{180} to \acid{M}{183}. A few additional contacts appear between F1 and the N-lobe, but these are only minor spots.
\clearpage
%
%
\section{Impact of the stabilising force}
\label{forceana}
In setup 3 and setup 4, we stabilised each protein with an external force acting on the FERM domain (see \autoref{motivation}). To ensure that this restraining does not cause major artefacts, the force-dependency of the used observables is examined below.\\
\\
The force has a mean value of $2.44\,\forceunit$ and a standard deviation of $21.80\,\forceunit$. Interestingly, the positive mean value shows that the proteins are rather pushed downwards than pulled up. This indicates that the stabilising force does not have to hold the protein upright the whole time. % TODO: clarify?
\\
Linear regressions show that all of the quantities $d_F$, $d_\text{F1-N}$, $d_\text{F2-C}$ and CA have a negligible correlation to the applied force. Here, negligible means that either the regression result was not significant or that the obtained slope was so small, that a change of one standard deviation in force would not change the quantity noticeably.\\
\\
We also tested the inter-residue distances for correlation with the applied force. To this end, 10 different proteins without neighbours were monitored, each for $1\,\si{\micro\second}$. For each residue pair in this dataset, we performed a linear regression. \autoref{force:contactmap} shows the calculated Pearson correlation coefficient (only significant correlations with Pearson $\left|r\right| > 0.3$). The mean value of the slope for the positively correlated pair distances is $33.7\,\si{\pico\newton/\nano\metre}$ and $-32.7\,\si{\pico\newton/\nano\metre}$ for the negatively correlated pair distances. Thus, the force can influence residue pairs contributing to the interface. However, we do not expect major changes in the contact regions, since the majority of residue pairs show only weak correlations.\\
\\
In summary, we do not expect large perturbations of our observables due to the force. However, this approach has limitations. First, there could be binding poses in multiple FAK interactions requiring a large inclination of the FAK molecule. These states would be suppressed by the force. Another limitation is that the force does not only prevent tilts around the long axis of FAK (falling to sideways), but also around the short axis which happens e.g. in FERM-FERM dimers. Lastly the reference to the z-axis is problematic. A reference to the membrane might be better, because it would involve membrane curvature as well.
%
%
%
\begin{figure}
	\centering
	\includegraphics[width=.7\textwidth]{figures/results/interface_corr}
	\nicecaption{Correlations in contact map}{The contact map shows correlations between inter-residue distances and the stabilising force. The mean slope is $33.2\,\si{\pico\newton/\nano\metre}$, the maximal Pearson $r < 0.5$.}
	\label{force:contactmap}
\end{figure}
%
%
%

\clearpage
%
%
\section{Conformational changes on a membrane}
\subsection{Conformational changes}
In this section the conformation of FAK bound to \pip{} (FAK-PIP) is compared to the observations from \autoref{sec:fak_sol} (FAK-SOL). For this purpose the simulation data of setup 4 was used with the condition, that other FAK molecules are more than $2\,\si{\nano\metre}$ away (0 neighbours). The contact map is based on the same dataset, which was used in \autoref{forceana:intramolec}.\\
\\
Analogously to \autoref{sec:fak_sol} the distribution of the COM distances is presented in \autoref{tobeadd} as an hexagonal binning plot. Again, different spots can be obtained. The spot with most encounters (spot 1) as well as the second spot (spot 2) are located at small values for both, $d_\text{F1-N}$ and $d_\text{F2-C}$. These spots show also smaller distances than FAK-SOL. In addition to this two spots appear, one at larger $d_\text{F2-C}$ (spot 3) and one at larger $d_\text{F1-N}$ (spot 4). These are however less populated and not as concentrated as spot 1 and 2. While spot 4 could be identified with spot 2 of FAK-SOL, spot 3 show completely new values for both COM distances.\\
\\
In \autoref{tobeadded} the difference of the contact maps of FAK-SOL and FAK-PIP can be found, again only for the interface. The distances between F2 and the C-lobe tend to get smaller, even if contacts around \acid{R}{665} show another trend. Also the contact between F1 and the N-lobe/activation loop show smaller distances. The RMSF values are increasing for all residue pairs.\\
Remarkable changes occur in the linker region. The residues around the autophosphorylation site \acid{Y}{397} increase their distances to residues \acid{M}{384} to \acid{S}{390} by up to $0.9\,\si{\nano\metre}$. Also the RMSF values are increased in the linker by up to $0.40\,\si{\nano\metre}$ near \acid{Y}{397}.\\
\\
With these observations the enhanced autophosphorylation of FAK bound to \pip{} can be comprehended. However the impacts on the interface between the FERM domain and the kinase are quit small. One reason for this could be the applied elastic network. Since electrostatics are treated with a cutoff radius, long range conformational changes have to be transferred along the residues. Therefore the choice of the correct force constant is of large importance to obtain allosteric effects in Martini.
\clearpage
%
%
\section{Multiple FAK interactions}
In this section the interactions occuring between multiple FAK molecules are analysed, for which the data from setup 4 is used. At this point the reader shall be reminded, that the used protein structure lacks the FAT domain, which is in full length FAK connected to the kinase via a linker region. This might make an important difference for clustering processes.
\subsection{Structure of FAK oligomers}
\label{mult:oligs}
The characterisation of the emerged FAK clusters is very difficult as they differ a lot in size and shape. The largest cluster observed in setup 4 had a size of 21 proteins, while there are other proteins, which did not join any cluster at all. Present shapes of the clusters include long chains as well as ring like conformations or just an agglomeration (see \autoref{tobeadded}). \\
\\
First of all the mean number of neighbours is examined. One can see in \autoref{mult:nng_vs_t} a fast rising in the number of neighbours in the beginning and a flattening after $6\,\si{\nano\second}$. The average of the five copies is at the end of the simulation $1.86$. This indicates a tendency to chains of FAK.\\
In \autoref{mult:inttype_vs_t} the average number of encounters of the different interaction types is plotted against the time. It shows, that FERM-kinase interactions (type 3) occur the most, while the others occur equally often. Only type 6 and type 7 (interactions, in which all four domains are involved) occur much less than the others.\\
From these observations one could draw the conclusion, that the preferred arrangement of FAK molecules is a chain, in which the FERM domain interacts with the kinase of the next molecule (FK-Chain). A second possibility would be a chain, in which the FERM domain interacts with the FERM domain of the next molecule, while the kinase interacts with the kinase of the previous molecule (FFKK-Chain). Assuming a FAK chain of length $n$, FK-Chains would show $n$ encounters of type 3 interactions, FFKK-Chains only $n/2$, but for both type 1 and type 2. This would also be consistent with the observed data.
%
%
%
\begin{figure}
	\subcaptionbox{Mean number of neighbours against the time\label{mult:nng_vs_t}. Filled area is observed minimum and maximum.}[0.49\textwidth]{
		\includegraphics[height=5cm]{figures/results/averagenumber}
	}\hfill%
	\subcaptionbox{Mean number of interactions against the time\label{mult:inttype_vs_t}. Filled area is observed minimum and maximum.}[0.49\textwidth]{
		\includegraphics[height=5cm]{figures/results/multiple_typevstime}
	}%
	\phantomsubcaption
\end{figure}
%
%
%
\subsection{Activation due to clustering}
At last the impacts of clustering on FAK activation are addressed. Activation means here the dissociation of the FERM domain from the kinase, therefore the obtained trajectories are analysed with respect to the contact area (CA) of the FERM-kinase interface as a key quantity. Unfortunately in no FAK molecule a full dissociation took place at any time, therefore only trends can be considered at this point.\\
%
%
%
\begin{figure}
	\centering
	\includegraphics[height=6cm]{figures/results/nng_ca}
	\captionof{figure}{CA for different number of neighbours}
	\label{mult:nng_ca}
\end{figure}
%
%
%
At first glance the CA seems to be independent of the number of neighbours a protein has as well of the interactions it is in (see explanatory for the number of neighbours \autoref{mult:nng_ca}). Therefore the data has to be filtered more.\\
Motivated from \autoref{mult:oligs}, only FAK molecules inside chains are taken into account. A FAK molecule can be seen as a chain member, if it has exactly two neighbours and if these neighbours are not neighbours of one another. For FK-Chain only type 3 interactions were allowed, for FFKK-Chains both, type 1 and type 2. The resulting distribution of CA as well as the COM distances $d_\text{F1-N}$ and $d_\text{F2-C}$ can be found in \autoref{mult:fk_ca} for FK-Chain and in \autoref{mult:ff_ca} for FFKK-Chain.\\
As one can see in the plots FK-Chains do not have an influence upon the CA. Also the distribution of $d_\text{F1-N}$ and $d_\text{F2-C}$ is very similar to the one obtained in \autoref{mem:comdist}, except for less sampling. However in FFKK-Chains the mean CA value is $2\,\si{\nano\metre}$ smaller than the one for FAK-MEM. Also the $d_\text{F2-C}$ seems to be populated more at larger values. Nevertheless all these changes are very small.
%
%
%
\begin{figure}
	\centering
	\includegraphics[height=6cm]{figures/results/fk_ca}
	\captionof{figure}{Analysis of the FERM-kinase interface in FK-Chains. The blue line was obtained from FAK-SOL.}
	\label{mult:fk_ca}
\end{figure}
%
%
%
%
%
%
%
\begin{figure}
	\centering
	\includegraphics[height=6cm]{figures/results/ff_ca}
	\captionof{figure}{Analysis of the FERM-kinase interface in FFKK-Chains. The blue line was obtained from FAK-SOL.}
	\label{mult:ff_ca}
\end{figure}
%
%
%
%


