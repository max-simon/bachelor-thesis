\section{FERM FERM dimerization}
As explained in \autoref{linktointroduction} FAK dimerizes via the FERM domains. These dimers have been observed in the simulations as well and are analysed below.\\ %TODO: that not a nice introduction!
% protein was stable, did not dissociate
From the trajectories one FERM FERM dimer was extracted with a total simulation time of $4.4\,\si{\micro\second}$. This restriction had to be done, because all other dimers existed only for less than $1\,\si{\micro\second}$, before a third protein started to interact with the dimer.\\
In the contact map of the interaction interface (see \autoref{tobeadded}) one can see, that only the F3 subdomains are interacting with each other. \acid{W}{266}, which is located in spot 1 and spot 2, is interacting with several residues from the other protein, which fits to the key role of this residue proposed by \textcite{fermdimer}. These spots have a mean distance of $0.6\,\si{\nano\metre}$. Also the residues \acid{G}{279} to \acid{T}{284} and \acid{N}{289} to \acid{T}{294} contribute to the interaction. Their mean distance is about $0.7\,\si{\nano\metre}$. However their RMSF values indicate, that these interactions are more flexible than these of \acid{W}{266}.\\
% conformational changes -> nope. Maybe the kinase is pressed into the FERM due to membrane?


