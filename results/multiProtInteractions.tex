In this section the interactions occuring between multiple FAK molecules are analysed, for which the data from setup 4 is used. At this point the reader shall be reminded, that the used protein structure lacks the FAT domain, which is in full length FAK connected to the kinase via a linker region. This might make an important difference for clustering processes.\\
The characterisation of the emerged FAK clusters is very difficult as they differ a lot in size and shape. The largest cluster observed in setup 4 had a size of 21 proteins, while there are other proteins, which did not join any cluster at all. Present shapes of the clusters include long chains as well as ring like conformations or just an agglomeration (see \autoref{tobeadded}). Nevertheless in none FAK molecule the FERM domain dissociated from the kinase, which means no activation took place.\\
\\
For further analysis interactions between two proteins are classified into 9 different interaction types (see \autoref{einbildsagtmehralstausendworte}). Two proteins are interacting, if their minimal distance is smaller than $1.5\,\si{\nano\metre}$ and are called neighbours in this case. In contrast a protein belongs to a cluster, if it has at least one neighbour inside the cluster. A single protein without neighbours has clustersize 1.\\
First of all the mean number of neighbours is examined. One can see in \autoref{tobeadd} a fast rising in the number of neighbours in the beginning and a flattening after $6\,\si{\nano\second}$. The average of the five copies is at the end of the simulation $1.86$. This indicates a tendency to chains of FAK.\\
In \autoref{tobeadd} the average number of encounters of the different interaction types is plotted against the time. It shows, that FERM-kinase interactions (type 3) occur the most, while the others occur equally often. Only type 6 and type 7 (interactions, in which all four domains are involved) occur much less than the others.\\
From these observations one could draw the conclusion, that the preferred arrangement of FAK molecules is a chain, in which the FERM domain interacts with the kinase of the next molecule.


%In this section the interactions, which occur in FAK clusters, are analysed. At this point the reader shall be reminded of the limitation of our model, i.e. the dropping of the FAT domain. In contrast to our model full length FAK includes also a FAT domain which is connected to the kinase via a linker region. Therefore the observed interactions might be very different from those of full length FAK.\\
%The characterisation of the emerged FAK clusters is very difficult as they differ a lot in size and shape. The largest cluster observed in the five trajectories had a size of 21 proteins while there are other proteins, which did not join any cluster at all. Both, closed conformation like rings, as well as long chains were observed, as you can see f.e. in \autoref{tobeadded}.\\
%It is worth to analyse the neighbouring types in the cluster. In \autoref{tobeadded} the average number of neighbours for the five trajectories is plotted against the time. Although there are clusters of more than 20 proteins, the number of neighbours stays below 2 (maximum of the five trajectories is 2.5). \autoref{tobeadded} shows the number of encounters of a give neighbour type as a function of time. These, in which only two domains interact (FERM-FERM, FERM-kinase and kinase-kinase), occur most often, especially type 3 (FERM-kinase) appears very often. These findings may suggest, that chains of multiple FAK connected as type 3 neighbours are a preferred arrangement. However this could be a relict of dropping the FAT domain or too short simulation times since auto arrangement of large molecules like proteins is a slow process.\\
%Clustering of FAK did not induced activation in the simulations. At any time the minimal distance between the FERM domain and the kinase of one protein was lower than $0.8\,\si{\nano\metre}$ and also long ranged configurational changes could not be observed.