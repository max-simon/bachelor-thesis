\subsection{Free energy profile of basic patch}
\label{results:umbrella}
In order to understand the falling of FAK onto the membrane the free energy profile of the basic patch \pip{} was investigated. Since this binding is the main contact between FAK and the membrane in our model, a short report on the results is given at this point.\\
The profiles are obtained from setup 2. The reaction coordinate is the z component of the COM distance between \pip{} and the protein fragment. For each forcefield, C36, \martini{} and \martini{} with PW, the average profile out of the five copies together with the standard deviation can be found in \autoref{umbrella:profiles}. The range $6\,\si{\nano\metre} \le z \le 7\,\si{\nano\metre}$ was set as zero point.
Both, C36 and \martini{}, show a similar energy depth of $\approx 17 \si{\kilo\joule/\mole}$ and a similar slope between $3\,\si{\nano\metre} \le z \le 4\,\si{\nano\metre}$ . Certainly, \martini{} shows systematically larger values than the all atom simulation. This can be attributed to the proposed underestimation of electrostatic forces due to the unpolar water beads (see \autoref{subsub:coarsegraining}). The difference at $z = 4.2\,\si{\nano\metre}$ originates from the different treatment of long range electrostatics: \martini{} uses a cut-off radius, C36 uses PME.\\ %TODO: kink in C36? Due to short range?
Also \martini{} with PW uses PME for long range electrostatic interactions and indeed it fits much better to the C36 profile for larger distances. However the binding strength is crucially underestimated in Martini with PW.\\
The extent to which \martini{} reproduces the results from all-atom simulations is remarkable, even though the parameters for \martini{} were obtained a.o. from free energy calculations (see \autoref{subsub:coarsegraining})\\
\\
In the used starting configurations, the proteins are already bound to \pip{}. Therefore a correct binding strength and the shape near the minimum is of larger interest than a correct sampling of farther distances. In addition \martini{} with PW required a much higher computational effort. That is why in the following simulations only \martini{} with the standard water model is considered.
%
%
%
\begin{figure}
	\centering
	\includegraphics[width=.8\textwidth]{figures/results/umbrella}
	\nicecaption{Free energy profile of basic patch}{For each forcefield the average and standard deviation of the five copies is presented.}
	\label{umbrella:profiles}
\end{figure}
%
%
%