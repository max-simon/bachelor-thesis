% TODOS!!!
%	-> write Section instead of section
%	-> rearrange images and paragraphs
%	-> check (see ...)
%   -> check Text.// bild // Text.
%	-> write in methods, dass sich Protein auch drehen kann, auch in Markov-Chain.
% kompiliert mit XeLaTeX
\documentclass[
11pt, % font size
parskip=half, % space between paragraphs instead of indentation
digital, % digital or print
oneside, % oneside / twoside
%	openright, % openright for title page and subsequent chapters on right pages in twosided layout
]{bsc}

\usepackage {xltxtra}
\usepackage{microtype}
% Package für Bilder
\usepackage{graphicx}
\usepackage{enumitem}

\title{Understanding Focal Adhesion Kinase Clustering Using Coarse-Grained MD Simulations}
\author{Max SImon}
\date{August 21, 2017}

% Mathe
\usepackage{amsmath, amssymb, booktabs, hyperref, subcaption, siunitx}

\bibliography{Paper}

% Code
\usepackage{listings}

\newcommand{\diff}{\text{d}}
\newcommand{\pip}{PI(4,5)P$_2$}
\newcommand{\acid}[2]{#1$^{#2}$}


\newcommand{\gromacs}{GROMACS}
\newcommand{\martini}{MARTINI}
\newcommand{\charmm}{CHARMM36}
\newcommand{\conan}{CONAN}

\newcommand{\pope}{POPE}
\newcommand{\popc}{POPC}
\newcommand{\forceunit}{\si{\pico\newton}}

% MyFig: \myFig[width]{source}{label}{figure title}{text}
\newcommand{\nicecaption}[2]{\caption[#1]{\small{\textbf{#1}\\#2}}}


%\includeonly{abstract, motivation}

\begin{document}
% abstract english
\chapter*{Abstract}
Focal adhesion kinase (FAK) is a tyrosine kinase associated to focal adhesions.

Mein erstes wort ist yo.
%	
%	
% chapter introduction
\chapter{Introduction}
Focal adhesions (FA) are macromolecular protein complexes which act as a connection hub between the cell, especially the cytoskeleton, and the extracellular matrix (ECM). They enable the cell to exert tension forces, but can also transduce mechanical stimuli from the ECM to the inside of the cell and integrate this information. One important protein associated to FAs is the focal adhesion kinase (FAK). FAK occurs in several signalling pathways and is a key player in integrating extracellular stimuli. It is of large interest not least because often an overexpression of FAK can be found in cancer cells, and understanding the activation processes and dynamics of FAK could give rise to new cancer treatments \autocite{cancerFAK}.
\section{Structure of FAK}
FAK consists of four domains: (i) a FERM (4.1 protein, ezrin, radixin and moesin) domain at the N-terminus, (ii) a tyrosine kinase, (iii) a proline-rich region and (iv) a focal adhesion targeting (FAT) domain at the C-terminus (see \autoref{intro:fak}).\\
FERM is a common protein domain which links proteins to membranes by binding to various phospholipids \autocite{fermdomain} and consists of three subdomains: F1, F2 and F3. In the F2 subdomain, a basic patch ($^{216}$KAKTLRK$^{222}$) can be found, which is a prominent binding site for phosphatidylinositol-4,5-bisphosphate (\pip).\\
The kinase consists of a C-lobe, an activation loop and an N-lobe. Catalytic activity of the kinase is mainly regulated by the phosphorylation of \acid{Y}{576}{} and \acid{Y}{577}, which are located in the activation loop \autocite{tyrosinePhosphor}. The kinase also provides binding sites for \pip{}. One is located next to the basic patch of the FERM domain, but others (namely \acid{R}{508}, \acid{R}{514}, \acid{K}{515}, \acid{K}{621} and \acid{K}{627}) can be found on the side of the kinase \autocites{pap002}{pap002Exp}.\\
The FERM domain and the kinase are connected by a linker region. In contrast to other kinase domains, the main autophosphorylation site of FAK, \acid{Y}{397}, can be found in this region and not in the kinase itself \autocite{pap001}.\\
The FAT domain is linked to the kinase by a flexible proline-rich region. FAT links to FAs by interacting with talin and paxillin which are integrin-associated proteins \autocite{fatdomain}.
%
%
%
\begin{figure}
	\centering
	\includegraphics[height=6cm]{figures/introduction/fak}
	\nicecaption{Structure of FAK}{The FERM domain, consisting of F1 (green), F2 (iceblue) and F3 (blue), and the kinase, consisting of N-lobe (violet), activation loop (black) and C-lobe (red), are connected via the linker region (white). The basic patch (in F2), \acid{Y}{397} (in the linker) and \acid{Y}{567}, \acid{Y}{577} (in the activation loop) are shown in atomistic representation.}
	\label{intro:fak}
\end{figure}
%
%
%
\section{\pip{} binding and activation}
It is known that integrin signalling due to cell adhesion triggers activation of FAK and that \pip{} is an important mediator in this signalling pathway. FAK interacts in several ways with \pip{}, which is discussed in the following.\\
\\
In the autoinhibited conformation, the FERM domain shields the active site of the kinase. Therefore, an activation of FAK requires the dissociation of the FERM and the kinase \autocite{structFAK}.\\
\pip{} is a phospholipid which is locally generated in FAs due to integrin signalling \autocite{pip2LocalGeneration}. Its charge depends strongly on the pH, but at physological conditions a net charge of -4 is the preferred state. However, in presence of basic amino acids (Arg, Lys), the deprotonated state gets promoted, resulting in a net charge of -5 \autocite{pip2_minus5}. The electrostatic binding of \pip{} to the basic patch in the F2 subdomain results in allosteric changes, which also influence the interface between the F1 subdomain and the N-lobe. Moreover, the linker region gets less strongly bound, which promotes autophosphorylation of \acid{Y}{397}. Because the \pip{} binding alone has no effect on the catalytic activity, which suggests that the FERM domain is still associated to the kinase \autocites{pap001}{pap003}. For activation, an additional stimulus, either biochemical or mechanical, is needed.\\
\\
Autophosphorylation of \acid{Y}{397} is an important step in FAK activation. Once this site is phosphorylated, it becomes a suitable binding site for SH2 or SH3, which are subdomains of proteins of Src family tyrosine kinases. Due to the conformational changes induced by \pip{}, this kinase has access to \acid{Y}{576} and \acid{Y}{577}. As said, the phosphorylation of these residues makes FAK fully active, resulting in dissociation of the FERM domain and the kinase \autocite{pap001}. Once the activation loop is phosphorylated, the FERM domain cannot inhibit the kinase any more. Therefore, a dephosphorylation is required to downregulate FAK activity \autocite{structFAK}.\\
\\
Mechanical forces can lead to dissociation of the FERM domain from the kinase as well. Forces acting on the FAT domain are transduced to the FERM-kinase interface because the linker is connected to the kinase, while the FERM domain is anchored onto a \pip{} containing membrane. These forces can lead to activation of FAK. In that way, it is acting as a mechanical sensor \autocite{pap004}.\\
\\
The binding sites for \pip{} on the side of the kinase were identified via computer simulations \autocite{pap002} and have been confirmed recently in experiments \autocite{pap002Exp}. The findings from \textcite{pap002Exp} show that these residues are required for catalytic activity of the kinase, and that they bind to phospholipids \textit{in vivo}. However, since the catalytic activity is not regulated by \pip{} alone, this binding was hypothesized to act as a stabilisation of the active state only \autocite{pap002Exp}.
\section{Dimerization and clustering}
\label{intro:clustering}
Experimental findings suggest that self-association of FAK, such as dimerization or clustering, can trigger autophosphorylation of \acid{Y}{397} \autocites{transAuto}{dimersVsClusters}.\\
\\
The FERM domain induces a dimerization of FAK by interacting with the FERM domain of a neighbouring FAK molecule. The interaction emerges around \acid{W}{266} in the connected FERM domains and is stabilised by an interaction of the FAT domain with the basic patch of the other FERM domain, respectively. \pip{} is not required for the dimerization of the FERM domains. However, an enriched FAK concentration is needed to observe FAK dimers in cells, which is the case at FAs. It is still unclear how the dimer could be stabilised at membranes, where the basic patch is also required for ligand binding \autocite{fakdimers}.\\
\\
\pip{} does not only induce conformational changes, but also clustering of several FAK molecules on the membrane \textit{in vitro} \autocite{pap001}. In contrast to dimers, clusters of FAK molecules can trigger additional biochemical stimuli \autocite{dimersVsClusters} and may play an important role in the scaffolding function of FAK for FAs \autocite{pap001}.

%
%
% chapter motivation
\chapter{Motivation}
As described in \autoref{intro:clustering}, intermolecular interactions of FAK molecules can effect their activation. The understanding of these interactions is of interest because they might play an important role in tumour cells in which FAK is often overexpressed. In addition, since FAK contributes to several signalling pathways and has a downstream effect on important cell functions, new insights into the activation process of FAK can be valuable for further studies as well.\\
The aim of this project is to get first atomistic insights into the clustering process of FAK molecules on a \pip{}-containing membrane. To this end, we let multiple FAK molecules aggregate on a membrane and analyse the clustering process and conformational changes in the molecules. With these observations, more specific investigations on interactions of multiple FAK molecules might become accessible.\\
\\
In this project, we used coarse-grained MD simulations with the \martini{} force field. \martini{} lacks chemical details, but it is a necessary simplification since systems with several FAK molecules involve a large number of particles and clustering processes have a large time scale.\\
Previous work in the group revealed problems in the use of \martini{} for simulations of FAK on a \pip{}-containing membrane, which are briefly discussed in \autoref{stabilising}. It was suggested that they occur due to an underestimation of the binding of the basic patch to \pip{} in \martini{}. Therefore, the examination of this binding was part of the project as well.

%
%
% chapter methods
\section{Methods}
The present thesis is meant to give an insight into dynamics and characteristics of FAK-FAK interactions as well as interactions with a \pip{} containing membrane obtained from MD simulations. For this purpose the Martini force field and C36 were used.\\
All simulations were made with starting configurations adapted from previous work in the group (C36 forcefield: \textcite{pap003}, Martini forcefield: \textcite{SARA}). Only a FERM-kinase fragment without the FAT domain (residues 35 to 686, PDB 2J0J \autocite{structFAK}) was used. As simulation options the recommended values were used and can be found IN THE APPENDIX. All simulations have been done at a temperature of 300$\si{\kelvin}$.\\
As explained in \autoref{subsub:coarsegraining} proteins have to be stabilized in Martini with elastic networks. This was set up by \textcite{SARA} to act only inside domains with a force constant of 830 $\si{\kilo\joule\mole^{-1}\nano\meter^{-2}}$. Therefore the interface between FERM domain and kinase is not affected and the linker is still flexible.
\subsection{Free energy of \pip{} binding}
In the first part of this thesis the free energy landscape of the \pip{} binding to the basic patch was retrieved for C36, Martini and Martini with PW using umbrella sampling.\\
For simplicity only a part of the F2 subdomain was used (residues 107 to 219). The F2 was placed above a \pip{} embedded into a phosphatidylethanolamine (\pope{}) membrane (see PICTURE). The whole system was transferred to a Martini structure with a provided transformation tool \autocite{backwardpy} and the elastic network was applied.\\
After a short equilibration the protein was pulled slowly away from the membrane using a distance pull between the COM of the protein and the COM of the \pip{}. This simulation was used to retrieve starting conformations for the umbrella windows (between 90 and 120, dependent on the resulting sampling). Afterwards each umbrella window was equilibrated for 0.5$\si{\nano\second}$ and then simulated for 6$\si{\nano\second}$ (10$\si{\nano\second}$ for Martini and Martini with PW). For each forcefield the pulling and umbrella sampling was done five times to estimate the statistical error.\\
The presented results are based on a total simulation time of 6.33$\si{\micro\second}$ for Martini, 5.64$\si{\micro\second}$ for Martini with PW and 3.88$\si{\micro\second}$ for C36. % C36: 1=180, 4.2=135, 5=102, 5.2=102, 6=127, Mart: 1=134, 2=126, 3=130, 4=113, 5=130, MartPol: alle=113
\subsection{Binding pose on membrane}
Reasons, why Saras can not be.
All atom and Polarizable water.

%
%
% chapter setup
\chapter{Setup}
\section{Protein structure}
All simulations have been done with starting configurations adapted from previous work in the group (C36 forcefield: \textcite{pap003}, \martini{} forcefield: \textcite{sara}). These configurations contain only a FERM-kinase fragment without the FAT domain and its linker (only residues 35 to 686, PDB 2J0J \autocite{structFAK}).\\
As explained in \autoref{subsub:coarsegraining} the secondary structure of proteins have to be stabilized in \martini{} using elastic networks. This was set up by \textcite{sara} to act only between backbone beads of the same domain which are within a cut-off radius of $1\,\si{\nano\metre}$. Therefore, the interface between FERM domain and kinase is not affected and the linker is still flexible. The force constant is 830 $\si{\kilo\joule\mole^{-1}\nano\meter^{-2}}$. 
\section{Setup 1 - FAK in solution}
\label{setup:setup1}
Setup 1 refers to a \martini{} simulation of a single FAK molecule in waterbox. NaCl ions were added to neutralize the charge (see \autoref{setup:setup1_pic}).\\
After a short equilibration the system was simulated for $20\,\si{\micro\second}$ at a temperature of $300\,\si{\kelvin}$. We used the default parameters of the \martini{} forcefield as input parameters.
%
%
%
\begin{figure}[h]
	\centering
	\includegraphics[height=6cm]{figures/setup/setup_free}
	\nicecaption{Setup 1 - FAK in solution}{The \martini{} structure of FAK (FERM blue, kinase red and linker black, 1486 beads) was put in a waterbox (ca. 50000 solvent beads) with ions (23 sodium beads and 20 chlorine beads). The box dimensions are $18.8\,\si{\nano\metre}$ x $18.8\,\si{\nano\metre}$ x $18.8\,\si{\nano\metre}$.}
	\label{setup:setup1_pic}
\end{figure}
%
%
%
\section{Setup 2 - Free energy of basic patch}
\label{setup:setup2}
For this setup only a part of F2 (residues 107 to 219, referred as F2 lobe in the following) was used. The lobe contains the basic patch and has a net charge of -5. Therefore no additional ions were needed.\\
We placed the lobe as a \charmm{} structure above a single \pip{} embedded into a phosphatidylethanolamine (\pope{}) membrane (see \autoref{setup:setup2_pic}). After a short equilibration, the F2 lobe was pulled slowly away from the membrane using a distance pull between the COM of the lobe and the COM of \pip{}. From this simulation, we retrieved starting conformations for the umbrella window. The number of umbrella windows was chosen accordingly to the sampling (between 90 and 120 windows). Each window was shortly equilibrated and afterwards simulated for $6\,\si{\nano\second}$.  From the trajectories of the umbrella windows the free energy profile was calculated using \gromacs{} WHAM implementation \autocite{gromacsWHAM}. The pulling and umbrella sampling was done five times to estimate the statistical error.\\
The starting configurations was transferred to a \martini{} structure (with both, standard water model and PW) with provided transformation tools \autocite{backwardpy}. Afterwards the elastic network was applied. Analogously to the simulation in \charmm{}, we retrieved starting configurations for the umbrella windows and simulated them for five independent copies. The simulation time for one umbrella window was increased to $10\,\si{\nano\second}$.\\
The presented results are based on a total simulation time of $3.88\,\si{\micro\second}$ for C36, $6.33\,\si{\micro\second}$ for \martini{}, $5.64\,\si{\micro\second}$ for \martini{} with PW. The temperature in the simulations was kept at $300\,\si{\kelvin}$. For all three force fields the default parameters were used.
% C36: 1=180, 4.2=135, 5=102, 5.2=102, 6=127, Mart: 1=134, 2=126, 3=130, 4=113, 5=130, MartPol: alle=113
%
%
%
\begin{figure}[h]
	\centering
	\includegraphics[height=6cm]{figures/setup/setup_umbrella}
	\nicecaption{Setup 2 - Free energy of basic patch}{The \charmm{} structure of the lobe (1936 atoms) above one \pip{} (143 atoms) embedded in a \pope{} membrane (509 lipids, ca. 63600 atoms). The box was filled up with ca. 63000 water molecules. The corresponding \martini{} structure contains 269 beads for the lobe, 6125 beads for the membrane and ca. 16200 solvent beads/PW molecules. The box dimensions are $14.0\,\si{\nano\metre}$ x $9.0\,\si{\nano\metre}$ x $20.0\,\si{\nano\metre}$.}
	\label{setup:setup2_pic}
\end{figure}
%
%
%
\section{Setup 3 - FAK on a \pip{} membrane}
\label{setup:setup3}
Setup 3 is a \martini{} simulation adopted from \textcite{sara}. It contains a single FAK molecule which was placed on a phosphatidylcholine (\popc{}) and \pip{} membrane (\pip{} concentration $15\%$). NaCl were added to neutralize the system (see \autoref{setup:setup3_pic}). In contrast to \textcite{sara}, we applied the stabilizing force explained in \autoref{motivation} to the protein.\\
Five independent copies were simulated for $10\,\si{\micro\second}$ each. The temperature was kept at $300\si{\kelvin}$ and the default parameters of \martini{} were used.
%
%
%
\begin{figure}[h]
	\centering
	\includegraphics[height=6cm]{figures/setup/setup_gen}
	\nicecaption{Setup 3 - FAK on a \pip{} membrane}{The setup involves one FAK molecule (FERM blue, kinase red and linker black, 1486 beads) on top of a \pip{} containing membrane (667 \pope{} lipids, 54 \pip{} lipids, 21918 membrane beads). The surrounding solution consists of ca. 21400 solvent beads and 473 ions. The box dimensions are $15.2\,\si{\nano\metre}$ x $15.2\,\si{\nano\metre}$ x $17.0\,\si{\nano\metre}$.}
	\label{setup:setup3_pic}
\end{figure}
%
%
%
\section{Setup 4 - FAK cluster}
From each copy in setup 3, we cut out five frames. All these 25 frames were arranged on a {5x5} grid (\autoref{setup:setup4_pic}). Each of the 25 proteins was stabilized with the external force independently. After a short equilibration the system was simulated for $9\,\si{\micro\second}$. We set up 25 different copies, regarding to the arrangement of the frames, resulting in a total simulation time of $45\,\si{\micro\second}$. The temperature was kept at $300\,\si{\kelvin}$ and the default parameters of \martini{} were used.
%
%
%
\begin{figure}[h]
	\centering
	\includegraphics[height=6cm]{figures/setup/setup_cluster}
	\nicecaption{Setup 4 - FAK cluster}{The setup is a combination of 25 frames from \autoref{setup:setup3}. It consists of ca. 807000 beads in a box of size $80.0\,\si{\nano\metre}$ x $80.0\,\si{\nano\metre}$ x $17.0\,\si{\nano\metre}$. The spaces between the frames were needed during setup to avoid an overlapping of the beads but disappears during the equilibration.}
	\label{setup:setup4_pic}
\end{figure}
%
%
%
%
%
% chapter results
\chapter{Results}
In this chapter, we present our results from the simulations. First of all FAK in solution is considered. Afterwards the focus is on the effect of \pip{}. We will start this section with the free energy profile of the basic patch. Since the simulations containing a membrane were carried out with a stabilizing force, we will examine the impact of the force on the observables, before we investigate the conformational changes of FAK induced by binding to the membrane. At the end the results obtained for multiple FAK molecules are presented.
%
%
\section{FAK in solution}
% TODO: link to methods
In this section the conformation of FAK in absence of \pip{} and other FAK molecules is analyzed. For this purpose the simulation data of setup 1 were used (see \autoref{setup:setup1}).\\
\\
First the COM distances of F1 to the N-lobe ($d_\text{F1-N}$) and F2 to the C-lobe ($d_\text{F2-C}$) are considered. In \autoref{tobeadded} a hexagonal binning plot of both values can be found, which indicates, that there are two different states: one in which $d_\text{F2-C}$ is larger (spot 1) and $d_\text{F1-N}$ smaller and one the other way around (spot 2). The systems starts in spot 1 and goes to spot 2 after $7\,\si{\micro\second}$ (this transition goes along with several frequent transitions), where it stays until the end of the simulation.\\
\textcite{jing} performed equivalent simulations in C36 and obtained
\begin{align}
	d_\text{F1-N} &= (3.30 \pm 0.40\,\text{std})\si{\nano\metre}\\
	d_\text{F2-C} &= (3.15 \pm 0.15\,\text{std})\si{\nano\metre}
\end{align}
These values can not be classified in one of the two spots as both distances are lower than those obtained in Martini.\\
\\
A contact map of the interface between the FERM domain and the kinase for frames of spot 2 can be found in \autoref{tobeadd}. Two contact areas can be identified. The first one (area 1) is located between F1 and the N-lobe/activation loop. It shows i.e. contacts between \acid{Y}{576} and \acid{Y}{577} and residues of the FERM domain. The minimal distance in this area, between residue \acid{H}{41} and \acid{Y}{576}, is $0.45\,\si{\nano\metre}$ with an RMSF value of $0.03\,\si{\nano\metre}$. This reflects the burying of the activity regulating residues in closed state.\\
The second contact area (area 2) is located between F2 and the C-lobe. The spots occur around the residues \acid{Y}{180} and \acid{D}{200} of F2 as well as \acid{F}{596} and \acid{R}{665}. The minimal distance in this area occurs between \acid{Y}{180} and \acid{F}{596} with $0.45\,\si{\nano\metre}$ and an RMSF value of $0.02\,\si{\nano\metre}$. These two residues have been reported as important actors in the interface by showing, that a mutation disturbs the interface and enhances the activation of FAK.\\
The linker show contacts with both domains. Interestingly the minimal distance in the marked areas (area 3 and area 4) occur between the autophosphorylation site \acid{Y}{397} and \acid{H}{58} ($0.45\,\si{\nano\metre}$, RMSF $0.03\,\si{\nano\metre}$) in F1 or \acid{Y}{576} ($0.50\,\si{\nano\metre}$, RMSF $0.10\,\si{\nano\metre}$) in the kinase. This is consistent with the concept, that autophosphorylation is prevented in closed conformation by a binding of the linker to the FERM domain.\\
\\
In contrast to \autoref{tobeadd}, the contact map for frames of spot 1 show less contacts between F2 and the C-lobe, i.e. around the mentioned residues \acid{Y}{180} to \acid{M}{183}. A few additional contacts appear between F1 and the N-lobe, but these are only minor spots.
\clearpage
%
%
\section{Free energy profile of basic patch}
From the umbrella simulations the free energy profile was retrieved using \gromacs{} WHAM. In \autoref{bild} the average of the five copies together with the standard deviation can be found for each forcefield. The range between 6$\si{\nano\metre}$ and 7$\si{\nano\metre}$ was set to 0.\\
As you can see in the plot, the results for C36 and Martini are very similar: both show a depth of $\approx 17.5 \si{\kilo\cal/\mole}$ and the same slope between 3$\si{\nano\metre}$ and 4$\si{\nano\metre}$. However it seems like the Martini profile is shifted to smaller z-distances. One reason for this is, that due to the coarse graining the COM distances alters (in our case the COM distance of Protein and \pip{} decreased by 0.1$\si{\nano\metre}$ after coarse graining). Beyond 4$\si{\nano\metre}$ there is a larger difference: while the free energy profile flattens out in C36, a kink can be found in Martini. This is explained by the different treatment of long range electrostatic interactions. For this Martini just uses a cut off radius, C36 uses PME.\\
Also Martini with PW uses PME for long range electrostatic interactions and indeed this fits much better to the profile observed in C36. However the binding strength is drastically underestimated when using PW. The same effect was observed by simulations of a single FAK on a \pip{} containing membrane. Here the kinase dissociated from the membrane, which is probably due to an underestimation of the \pip{} binding site in the kinase.\\
The results indicate, that Martini is a suitable simplification for simulating FAK on membranes, which allow much longer simulations of much larger systems.
\clearpage
%
%
\section{Impact of stabilising force}
\label{forceana}
In setup 3 and setup 4, we stabilised each protein with an external force acting on the FERM domain (see \autoref{motivation}). To ensure that this restraining does not cause major artefacts, the force-dependency of the used observables is examined below.\\
\\
The force has a mean value of $2.44\,\forceunit$ and a standard deviation of $21.80\,\forceunit$. Interestingly, the positive mean value shows that the proteins are rather pushed downwards than pulled up. This indicates that the stabilising force does not have to hold the protein upright the whole time. % TODO: clarify?
\\
Linear regressions show that all of the quantities $d_F$, $d_\text{F1-N}$, $d_\text{F2-C}$ and CA have a negligible correlation to the applied force. Here, negligible means that either the regression result was not significant or that the obtained slope was so small, that a change of one standard deviation in force would not change the quantity noticeably.\\
\\
We also tested the inter-residue distances for correlation with the applied force. To this end, 10 different proteins without neighbours were monitored, each for $1\,\si{\micro\second}$. For each residue pair in this dataset, we performed a linear regression. \autoref{force:contactmap} shows the calculated Pearson correlation coefficient (only significant correlations with Pearson $\left|r\right| > 0.3$). The mean value of the slope for the positively correlated pair distances is $33.7\,\si{\pico\newton/\nano\metre}$ and $-32.7\,\si{\pico\newton/\nano\metre}$ for the negatively correlated pair distances. Thus, the force can influence residue pairs contributing to the interface. However, we do not expect major changes in the contact regions, since the majority of residue pairs show only weak correlations.\\
\\
In summary, we do not expect large perturbations of our observables due to the force. However, this approach has limitations. First, there could be binding poses in multiple FAK interactions requiring a large inclination of the FAK molecule. These states would be suppressed by the force. Another limitation is that the force does not only prevent tilts around the long axis of FAK (falling to sideways), but also around the short axis which happens e.g. in FERM-FERM dimers. Lastly the reference to the z-axis is problematic. A reference to the membrane might be better, because it would involve membrane curvature as well.
%
%
%
\begin{figure}
	\centering
	\includegraphics[width=.7\textwidth]{figures/results/interface_corr}
	\nicecaption{Correlations in contact map}{The contact map shows correlations between inter-residue distances and the stabilising force. The mean slope is $33.2\,\si{\pico\newton/\nano\metre}$, the maximal Pearson $r < 0.5$.}
	\label{force:contactmap}
\end{figure}
%
%
%

\clearpage
%
%
\section{Conformational changes on a membrane}
\subsection{Conformational changes}
In this section the conformation of FAK bound to \pip{} (FAK-PIP) is compared to the observations from \autoref{sec:fak_sol} (FAK-SOL). For this purpose the simulation data of setup 4 was used with the condition, that other FAK molecules are more than $2\,\si{\nano\metre}$ away (0 neighbours). The contact map is based on the same dataset, which was used in \autoref{forceana:intramolec}.\\
\\
Analogously to \autoref{sec:fak_sol} the distribution of the COM distances is presented in \autoref{tobeadd} as an hexagonal binning plot. Again, different spots can be obtained. The spot with most encounters (spot 1) as well as the second spot (spot 2) are located at small values for both, $d_\text{F1-N}$ and $d_\text{F2-C}$. These spots show also smaller distances than FAK-SOL. In addition to this two spots appear, one at larger $d_\text{F2-C}$ (spot 3) and one at larger $d_\text{F1-N}$ (spot 4). These are however less populated and not as concentrated as spot 1 and 2. While spot 4 could be identified with spot 2 of FAK-SOL, spot 3 show completely new values for both COM distances.\\
\\
In \autoref{tobeadded} the difference of the contact maps of FAK-SOL and FAK-PIP can be found, again only for the interface. The distances between F2 and the C-lobe tend to get smaller, even if contacts around \acid{R}{665} show another trend. Also the contact between F1 and the N-lobe/activation loop show smaller distances. The RMSF values are increasing for all residue pairs.\\
Remarkable changes occur in the linker region. The residues around the autophosphorylation site \acid{Y}{397} increase their distances to residues \acid{M}{384} to \acid{S}{390} by up to $0.9\,\si{\nano\metre}$. Also the RMSF values are increased in the linker by up to $0.40\,\si{\nano\metre}$ near \acid{Y}{397}.\\
\\
With these observations the enhanced autophosphorylation of FAK bound to \pip{} can be comprehended. However the impacts on the interface between the FERM domain and the kinase are quit small. One reason for this could be the applied elastic network. Since electrostatics are treated with a cutoff radius, long range conformational changes have to be transferred along the residues. Therefore the choice of the correct force constant is of large importance to obtain allosteric effects in Martini.
\clearpage
%
%
\section{Multiple FAK interactions}
In this section the interactions occuring between multiple FAK molecules are analysed, for which the data from setup 4 is used. At this point the reader shall be reminded, that the used protein structure lacks the FAT domain, which is in full length FAK connected to the kinase via a linker region. This might make an important difference for clustering processes.
\subsection{Structure of FAK oligomers}
\label{mult:oligs}
The characterisation of the emerged FAK clusters is very difficult as they differ a lot in size and shape. The largest cluster observed in setup 4 had a size of 21 proteins, while there are other proteins, which did not join any cluster at all. Present shapes of the clusters include long chains as well as ring like conformations or just an agglomeration (see \autoref{tobeadded}). \\
\\
First of all the mean number of neighbours is examined. One can see in \autoref{mult:nng_vs_t} a fast rising in the number of neighbours in the beginning and a flattening after $6\,\si{\nano\second}$. The average of the five copies is at the end of the simulation $1.86$. This indicates a tendency to chains of FAK.\\
In \autoref{mult:inttype_vs_t} the average number of encounters of the different interaction types is plotted against the time. It shows, that FERM-kinase interactions (type 3) occur the most, while the others occur equally often. Only type 6 and type 7 (interactions, in which all four domains are involved) occur much less than the others.\\
From these observations one could draw the conclusion, that the preferred arrangement of FAK molecules is a chain, in which the FERM domain interacts with the kinase of the next molecule (FK-Chain). A second possibility would be a chain, in which the FERM domain interacts with the FERM domain of the next molecule, while the kinase interacts with the kinase of the previous molecule (FFKK-Chain). Assuming a FAK chain of length $n$, FK-Chains would show $n$ encounters of type 3 interactions, FFKK-Chains only $n/2$, but for both type 1 and type 2. This would also be consistent with the observed data.
%
%
%
\begin{figure}
	\subcaptionbox{Mean number of neighbours against the time\label{mult:nng_vs_t}. Filled area is observed minimum and maximum.}[0.49\textwidth]{
		\includegraphics[height=5cm]{figures/results/averagenumber}
	}\hfill%
	\subcaptionbox{Mean number of interactions against the time\label{mult:inttype_vs_t}. Filled area is observed minimum and maximum.}[0.49\textwidth]{
		\includegraphics[height=5cm]{figures/results/multiple_typevstime}
	}%
	\phantomsubcaption
\end{figure}
%
%
%
\subsection{Activation due to clustering}
At last the impacts of clustering on FAK activation are addressed. Activation means here the dissociation of the FERM domain from the kinase, therefore the obtained trajectories are analysed with respect to the contact area (CA) of the FERM-kinase interface as a key quantity. Unfortunately in no FAK molecule a full dissociation took place at any time, therefore only trends can be considered at this point.\\
%
%
%
\begin{figure}
	\centering
	\includegraphics[height=6cm]{figures/results/nng_ca}
	\captionof{figure}{CA for different number of neighbours}
	\label{mult:nng_ca}
\end{figure}
%
%
%
At first glance the CA seems to be independent of the number of neighbours a protein has as well of the interactions it is in (see explanatory for the number of neighbours \autoref{mult:nng_ca}). Therefore the data has to be filtered more.\\
Motivated from \autoref{mult:oligs}, only FAK molecules inside chains are taken into account. A FAK molecule can be seen as a chain member, if it has exactly two neighbours and if these neighbours are not neighbours of one another. For FK-Chain only type 3 interactions were allowed, for FFKK-Chains both, type 1 and type 2. The resulting distribution of CA as well as the COM distances $d_\text{F1-N}$ and $d_\text{F2-C}$ can be found in \autoref{mult:fk_ca} for FK-Chain and in \autoref{mult:ff_ca} for FFKK-Chain.\\
As one can see in the plots FK-Chains do not have an influence upon the CA. Also the distribution of $d_\text{F1-N}$ and $d_\text{F2-C}$ is very similar to the one obtained in \autoref{mem:comdist}, except for less sampling. However in FFKK-Chains the mean CA value is $2\,\si{\nano\metre}$ smaller than the one for FAK-MEM. Also the $d_\text{F2-C}$ seems to be populated more at larger values. Nevertheless all these changes are very small.
%
%
%
\begin{figure}
	\centering
	\includegraphics[height=6cm]{figures/results/fk_ca}
	\captionof{figure}{Analysis of the FERM-kinase interface in FK-Chains. The blue line was obtained from FAK-SOL.}
	\label{mult:fk_ca}
\end{figure}
%
%
%
%
%
%
%
\begin{figure}
	\centering
	\includegraphics[height=6cm]{figures/results/ff_ca}
	\captionof{figure}{Analysis of the FERM-kinase interface in FFKK-Chains. The blue line was obtained from FAK-SOL.}
	\label{mult:ff_ca}
\end{figure}
%
%
%
%


%
%
% chapter conclusion
\chapter{Conclusion}
The aim of the present work was to gain insight into conformational changes of FAK due to \pip{} binding and interactions between multiple FAK molecules. To this end, we used coarse-grained MD simulations with the \martini{} force field.\\
\\
Previous work in the group revealed unnatural behaviour of FAK bound to a \pip{}-containing membrane in \martini{} simulations, namely a falling sidewards. We were able to proof that it is not induced by an underestimation of \pip{} binding at the basic patch. Indeed, \martini{} was able to reproduce the free energy profile of this binding obtained from a reference simulation in \charmm{} remarkably. However, the cause of the falling is still an outstanding question and should be addressed in further studies.\\ % FDA?
\\
In this project we introduced a workaround, namely a stabilising force acting on the FERM domain of the FAK molecules. In \autoref{forceana}, we showed that this approach doesn't influence the observables used in the remaining part, but we are aware if its limitations (discussed in \autoref{forceana}) and still investigate possible alternatives, for example flat-bottom potentials or cylindrical pulling \autocite[p. 156-158]{gromacsManual}.\\
\\
From the configurations obtained for FAK in solution, we identified important residues contributing to the FERM-kinase interface. The observations fit well with experimental studies of \textcite{structFAK}. Also the burying of the active site of the kinase as well as the hiding of the autophosphorylation site was observed in the simulations.\\
We compared these results to conformations obtained in FAK molecules bound to a \pip{}-containing membrane and identified configurational changes. They involve not only the promotion of the autophosphorylation site, but also a partial opening of the FERM-kinase interface. These changes are consistent with previous studies on allosteric effects of \pip{} binding to FAK by \textcite{pap001} and \textcite{pap003}.\\
\\
In \autoref{multiProt} we investigated the interactions between multiple FAK molecules on a membrane. We were able to confirm the importance of \acid{W}{266} in FERM-FERM dimerization proposed by \textcite{fakdimers}. However, our observations imply that FERM-kinase dimers form more often than FERM-FERM dimers.\\
Regarding more than two FAK molecules, we observed a tendency to aggregate in chain-like structures. These structures were investigated on activation-promoting features, like increased domain distances or a smaller contact area, but no significant differences have been observed.\\
\\
Because FERM-kinase dimers were observed the most, a further investigation on this interaction, including the identification of key residues and preferred binding poses, might contribute to the understanding of FAK clustering processes. In addition, the impact of the \pip{} concentration on FAK clustering should be addressed, since \pip{} concentration is linked to integrin signalling.





A further investigation of FAK dimers, especially FERM-kinase dimers, might 
Further studies

% TODO: changing pip2 concentration, alternatives to force, dimer interface in more detail
% TODO: classify instead of cluster, key residues, preferred poses.
Currently we try to cluster the obtained configurations with more general approaches in order to reveal the parameters inducing configurational changes in multiple FAK interactions. However, arrangement of large molecules such as FAK is a time consuming process, which has not come to an end in our simulations. Longer simulation times could therefore give new insights into the consequences of FAK clustering. 
%
%
%
\printbibliography[title=References]{}
% bibliography
\end{document}