% kompiliert mit XeLaTeX
\documentclass[
11pt, % font size
parskip=half, % space between paragraphs instead of indentation
digital, % digital or print
oneside, % oneside / twoside
%	openright, % openright for title page and subsequent chapters on right pages in twosided layout
]{bsc}

\usepackage {xltxtra}
\usepackage{microtype}
% Package für Bilder
\usepackage{graphicx}
\usepackage{enumitem}

\title{Understanding}
\author{Max SImon}
\date{August 21, 2017}

% Mathe
\usepackage{amsmath, amssymb, booktabs, hyperref, subcaption, siunitx}

\bibliography{../Paper}

% Code
\usepackage{listings}

\newcommand{\diff}{\text{d}}
\newcommand{\pip}{PI(4,5)P$_2$}
\newcommand{\acid}[2]{#1$^{#2}$}


\newcommand{\gromacs}{GROMACS}
\newcommand{\martini}{MARTINI}
\newcommand{\charmm}{CHARMM36}

\newcommand{\pope}{POPE}
\newcommand{\popc}{POPC}
\newcommand{\forceunit}{\si{\pico\newton}}

% MyFig: \myFig[width]{source}{label}{figure title}{text}
\newcommand{\nicecaption}[2]{\caption[#1]{\small{\textbf{#1}~#2}}}


\begin{document}
	
\chapter{Introduction}
\section{Focal adhesion kinase}
% TODO: scaffolding function of FAK, clustering
% TODO: more about cluster shape and so on
Focal adhesions (FA) are macromolecular protein complexes which act as a connection hub between the cell, i.e. the cytoskeleton, and the extracellular matrix (ECM). They enable the cell to performing tension forces, but can also trigger mechanical stimuli from the ECM. One important protein associated to FA is the focal adhesion kinase (FAK). FAK occurs in several signalling pathways and is a key player in integrating extracellular stimuli. It is of large interest not least because in cancer cells often an overexpression of FAK can be found and understanding the activation processes and dynamics of FAK could give rise to new cancer treatments.% \autocite{cancerFAK}.
\subsection{Structure}
FAK consists of four major domains (see \autoref{TOBEADDED}): a FERM domain as N-terminal, a tyrosine kinase, a proline rich region and a focal adhesion targeting (FAT) domain as C-terminal.\\
FERM (4.1 protein, ezrin, radixin and moesin) is a common protein domain, which targets proteins to membranes \autocite{fermdomain} and consists of three subdomains F1, F2 and F3. In the F2 subdomain there is the basic patch  ($^{216}$KAKTLRK$^{222}$), which is a prominent binding site for phosphatidylinositol-4,5-bisphosphate (\pip). This phospholipid is locally generated in FA due to integrin signaling \autocite{pap001}.\\
The kinase domain of the C-lobe, the activation loop and the N-lobe. The catalytic activity of kinase is regulated by the phosphorylation of \acid{Y}{576}\,and \acid{Y}{577}, which are located in the activation loop \autocite{tyrosinePhosphor}. The C-lobe also provides a binding site for \pip\,which is located next to the basic patch of the FERM domain \autocite{pap002}.\\
The FERM domain and the kinase are connected by a linker region. In contrast to other kinase domains the main autophosphorylation site of FAK \acid{Y}{397} can be found in this region \autocite{pap001}.\\ %TODO: better cite
The FAT domain is linked to the kinase by a flexible proline rich region. FAT targets to FA by interacting with talin and paxillin, which are proteins associated with FA \autocite{fatdomain}.
\subsection{Autophosphorylisation and activation}
A maximum catalytic turnover requires \acid{Y}{576}\,and \acid{Y}{577} in the activation loop to be phosphorylated. In the inactive state this region is shielded by the FERM domain. Also the autophosphorylation site \acid{Y}{397} is isolated by the FERM domain. Therefore an activation is only possible if the FERM domain dissociate at least partly from the kinase \autocite{structFAK}. FAK triggers several stimuli, but in this thesis the main focus lies on the allosteric effect of \pip\,binding to the basic patch.\\
% quelle -5: PIP2 and proteins: in- teractions, organization, and information flow
\pip\,is a has a net charge of -4, but in presence of K the deprotonated state gets promoted resulting in a net charge of -5. The electrostatic binding of \pip\,to the basic patch in the F2 subdomain results in long ranged configurational changes, which also influence the interface between the F1 subdomain and the N-lobe. Also the linker region gets less stronger bound so that an autophosphorylisation of \acid{Y}{397} is promoted. Phosphorylated \acid{Y}{397} is a suitable binding site for SH2, a subdomain of proteins from the Src kinase family. The presence of a kinase and the partial opening of the FERM kinase interface leads to a phosphorylisation of \acid{Y}{576}\,and \acid{Y}{577}. The resulting Src-FAK complex can act as an fully active kinase. In this state the FERM domain dissociates from the kinase \autocites{pap003}{pap001}.
\subsection{Dimerization and clustering}
%%%%%%%%%%%%%
% TODO: opening of interface , clustering, characteristics of the clustering
%%%%%%%%%%%%%
The FERM domain induces a dimerization of FAK as it is doing in other proteins containing a FERM domain as well. The interaction emerges around \acid{W}{266} in the connected domains and is stabilised by an interaction of the FAT domain with the basic patch of the other FERM domain respectively \autocite{fakdimers}. It has been shown, that the autophosphorylation happens in trans \autocite{transAuto} and requires \acid{W}{266} \autocite{fakdimers}. On a membrane however the dimer can not be stabilised by the FAT-FERM interaction, because the basic patch binds to \pip\,in the membrane, so the membrane has to stabilize the dimer. Besides dimers larger clusters of FAK can emerge, which can contribute to additional signaling processes \autocite{dimersVsClusters}.

\chapter{Motivation}
\label{motivation}
As described in \autoref{intro:clustering} the process of FAK clustering and its effect upon activation, especially on an atomistic scale, are still not understood and part of current research. In this thesis the results of MD simulations with the Martini force field (see \autoref{subsub:coarsegraining}) are presented, which address the clustering process of FAK molecules. In this context Martini is a necessary simplification due to the large number of particles in systems containing sever FAK molecules.\\
\\
However previous work in the group \autocite{sara} obtained difficulties in the use of Martini for simulations of FAK on a \pip{} containing membrane. In some simulations equivalent to setup 3 (see \autoref{setup:setup3}) except for an external force the protein rapidly changed its inclination to the membrane. In the following part this shall be characterised by the angle $\beta$ between the z-axis and the vector connecting F1 and F2, $\vec{d}_F$.
\begin{equation}
\cos\left(\beta\right) = \frac{\vec{d}_{F, z}}{d_F},\quad d_F = \left|\vec{d}_F\right|
\end{equation}
\textcite{sara} simulated five different copies for $10\,\si{\micro\second}$ each. The resulting distributions of the angle can be seen in \autoref{motiv:sarascurves}. Exemplary the red line shows a mean value of $90\,\si{\deg}$, which is what meant with a falling of FAK in the following part. The angle changed in less than XXX and stayed constant for the remaining simulation time.\\
There are several reasons why this is rather an artefact of the Martini force field than a possible binding pose of FAK to the membrane as suggested by \textcite{pap002}. The first one is, that FAK falls to both extensive sites, which means that the key residues for an interaction of the extensive site of the kinase with the membrane proposed by \textcite{pap002} are located on top of the FAK and not on the side of the membrane. Indeed contact analysis showed, that virtually all residues on the surface (in both, FERM domain and kinase) were interacting with the membrane. Another one is, that this behaviour was not observed in equivalent all atom simulations in C36 ($1.5\,\si{\micro\second}$ in total). Here only two maxima were observed around $8\,\si{\deg}$ and around $20\,\si{\deg}$, the largest observed angle was $40\,\si{\deg}$.\\
\\
In the course of this project several efforts were made to understand the cause of this falling, e. g. to review the binding of the basic patch in Martini, which is presented in \autoref{results:umbrella}. However the reason could not have proven beyond doubt. In order to still perform reasonable simulations of multiple FAK an external force was applied to each FAK molecule. This is called stabilizing force in the following parts.\\
The force is acting onto F1 and F2 parallel to the z-axis and is proportional to the deviation of their z-distance $\Delta z$ from a reference distance $z_0$. An illustration of the force can be found in \autoref{motiv:forceillustr}. For the determination of $z_0$ only the green and the blue distribution from \autoref{motiv:sarascurves} were considered, because the large angles observed in the other distributions have not been observed in C36 simulations. The mean value of $\vec{d}_{F, z}$ for these two distributions is $2.228\,\si{\nano\metre}$, which was therefore set as $z_0$. %TODO: get force constant out of distribution
\textit{@Csaba: sorry, but how did we get this force constant of 1000 out of the distributions?}\\
\\
%
%
%
\begin{figure}
	\subcaptionbox{\label{motiv:sarascurves}}[0.49\textwidth]{
		\includegraphics[height=5cm]{figures/introduction/sara_angles}
	}\hfill%
	\subcaptionbox{\label{motiv:forceillustr}}[0.49\textwidth]{
		\includegraphics[height=5cm]{figures/introduction/forceapproach}
	}%
	\nicecaption{Inclination angle of FAK}{(\subref{motiv:sarascurves}): Distributions of $\beta$ obtained by \textcite{sara}. (\subref{motiv:forceillustr}): Illustration of the applied force.}
\end{figure}
%
%
%
In the following two chapters the used methods, i.e. MD simulation, are explained and the used setups introduced. Afterwards the obtained results are presented. For this purpose FAK in solution and FAK bound to \pip{} are analysed and compared to known information from experiments or other simulations. Also the impacts of the stabilizing force onto the simulations are commented. At last the focus is set on interactions between multiple FAK molecules.


\chapter{Methods}
\section{Molecular dynamic simulation}
\section{Molecular Dynamics simulations}
Molecular dynamics (MD) simulations are an important tool to study biological systems on a microscopic scale. In the framework of MD simplifications are made, which enables simulations of several nanometres length scales for microseconds. Therefore MD is suitable for membranes, proteins and other biomolecules.\\
Because in this thesis \gromacs\,(Groningen Machine for Chemical Simulations) \autocites{gromacs1, gromacsManual} was used as MD engine, the following part refers to \gromacs\,conventions and features.
\subsection{The physics behind MD}
In MD the system is simulated on an atomistic scale but with classical mechanics only. In order to do so atoms are only treated as spheres without distinction into electrons and nuclei. Quantum mechanical (QM) effects, such as excitation or electron transfer processes, are therefore not accessible and neglected in MD. However the atoms are parametrized by effective parameters, which can be motivated from QM \autocite[p. 127f]{greenBook}.\\
\\
Newtons equation of motion
\begin{equation}
\vec{F}_i = m_i\frac{\diff^2 \vec{r}_i}{\diff t^2}
\end{equation}
where $\vec{F}_i$ is the force acting on particle $i$, $m_i$ its mass and $\vec{r}_i$ its position, can be turned into two first-order differential equations
\begin{align}
\label{eq:eq_of_mot_first_order}
\frac{\diff \vec{r}_i}{\diff t} = \vec{v}_i \\
m_i \frac{\diff \vec{v}_i}{\diff t} = \vec{F}_i
\end{align}
They can be integrated numerically with e.g. Leapfrog- or Verlet integration scheme. Both are low order, i.e. the integration error is $\mathcal{O}(\Delta t^3)$, which saves computational cost. However they have the great advantage to be time reversible and symplectic. Later means, that it conserves the phase space volume, like the Hamiltonian operator would do as well. Therefore the long term error in energy conservation stays small \autocite[p. 72ff]{UnderstandingMD}.\\
\\
The force acting on particle $i$ is given by the potential at its position.
\begin{equation}
\vec{F}_i = - \frac{\partial V}{\partial \vec{r}_i}
\end{equation}
In MD bonded and non-bonded interactions contribute to the potential $V$, but it is also possible to apply external forces.
\subsubsection{Bonded interactions}
Bonded interactions act intra molecular and describe chemical bonds. They can occur between two, three or four particles.\\
\\
An interaction between two bonded particles refers to their bond length. A deviation from the equilibrium bond length results in potential energy, which is usually described by an harmonic oscillator.
\begin{align}
V_{\text{dist, bond }i} = \frac{k_{\text{dist}}}{2}\left( r - r_{0} \right)^2
\end{align}
$k_{\text{dist,}}$ is the force constant and $r_0$ is the equilibrium bond length for the bond type of bond $i$. For larger deviations a Morse potential (exponential decaying potential for large deviations) is more precise, but has a much higher computational cost.\\
\\
Also a deviation from an equilibrium angle between three bonded partners, i.e. the bond angle, results in potential energy. A common description is the harmonic oscillator as well.
\begin{equation}
V_\text{angle} = \frac{k_\text{angle}}{2}\left( \theta - \theta_0 \right)^2
\end{equation}%TODO: write what is what or is this clear?
\\
The dihedral angle is the angle between two particles, which are separated by three bonds and can therefore be understood as the torsion angle of the intermediate two particles and bond. Besides a description of torsion this angle can also be used to preserve plane rings and the chirality of four particle groups. It is usually approximated with a periodic approach
\begin{equation}
V_{\text{dihedral, periodic}} = \frac{k_\text{dihedral}}{2}\left(1 + \cos\left(n \phi - \phi_0 \right)\right)
\end{equation}
where $k_\text{dihedral}$ describes the energy barrier for turning the dihedral angle, $n$ the number of minima in the energy function (multiplicity) and $\phi_0$ a phase factor \autocite[p. 71-83]{gromacsManual}.
\subsubsection{Non-bonded interactions}
Non-bonded interactions are present between all atoms in the system and act pairwise. In MD Pauli repulsion, van der Waals (vdW) forces and electrostatic forces are taken into account. Because bonded interactions use effective parameters they are usually excluded from non-bonded interactions.\\
\\
The Lennard-Jones potential combines Pauli repulsion ($r^{-12}$ term) and the vdW force ($r^{-6}$ term).
\begin{equation}
V_\text{Lennard-Jones} = \sum_{\text{non-bonded pairs i,j}} 4 \epsilon\left(\left(\frac{\sigma}{r_{ij}}\right)^{12} - \left(\frac{\sigma}{r_{ij}}\right)^6\right)
\end{equation}
$\epsilon$ is related to the potential depth and $\sigma$ to the potential range.\\
\\
The Coulomb potential is given by
\begin{equation}
V_\text{Coulomb} = \frac{q_1 q_2}{4 \pi \epsilon_0 \epsilon_r r}
\end{equation}
where $q_1$, $q_2$ are the (partial) charges of the interacting particles, $r$ their distance and $\epsilon_r$ the relative dielectric constant \autocite[p. 65-71]{gromacsManual}.\\
\\
In general non-bonded interactions act between all atoms in the system, which come along with a very large computational cost.\\
The easiest solution is to use a cut-off radius $r_c$. Particles behind this radius are not taken into account. This can be implemented very efficiently with Verlet neighbour lists. For each particle a neighbour list is created, which contains all particles inside a second radius $r_v$ with $r_v > r_c$. For a force calculation only the distances to the particles, which are part of the list, have to be calculated. The lists are updated, if the maximal displacement in the system is larger than $r_v - r_c$. This method is suitable especially for Lennard-Jones potential, because of its rapid decay $r_c$ can be chosen very small \autocite[p. 144]{greenBook}.\\
The electrostatic potential is proportional to $1/r$, that is why the use of a cut-off radius would lead to large jumps in the potential. Long range interactions have to be considered, which can be effectively done with Particle Mesh Ewald (PME) summation \autocite{pme}. Particle mesh methods in general split the electrostatic potential up into a short range and a long range part via a switching function. The short range part can be calculated with a small cut-off radius in real space. The long range part however is calculated by solving the Poisson equation of the actual charge distribution, for which a discrete grid (mesh) is used. In PME this grid is transformed to Fourier space, where the solution of the Poisson equation is a sum over the gridpoints. This requires of course periodic boundary conditions (see \autoref*{subsec:pbc}) \autocite[p. 246-251]{greenBook}.
\subsubsection{External forces}
In addition to bonded and non-bonded interactions external or artificial forces can be applied, such as pulling forces, restrains and constrains.\\
\\
With \gromacs\,it is possible to perform pulling forces onto groups of atoms in the system. In this thesis pulling was used to bias distances between groups. For this \gromacs\,provides an option to apply an umbrella potential to two groups which yields in a force, which is proportional to the deviation of the distance between the groups from a reference distance. The force can be applied on one or two spatial dimensions only or along a predefined vector and the reference distance can change during in time  \autocite[p. 154-159]{gromacsManual}.\\
Restrains are artificial potentials applied to positions, distances, angles, dihedral angles or orientations of particle groups to conserve labile configurations or to take additional information from experimental data into account \autocite[p. 84f]{gromacsManual}.\\
Constrains are set to keep properties, such as bond lengths constant. In order to do so, they get reset to the desired value after each time step \autocite[p. 44f]{gromacsManual}. % Lincs, shake?
\subsection{Forcefields}
The parameters for the potentials described above are provided by force fields. There is a wide range of force fields, which are optimized for different application fields. In this thesis tow different ones, CHARMM36 and Martini, are introduced.\\
\\
Force fields define specific atomtypes, which allow a mapping of atoms (particles in physical system) onto beads (particles used in the simulation). This mapping can take the environment of the particle (binding partners, solvents, nearby charges a.o.) into account, but can also neglect details by f.e. mapping several atoms to one bead (coarse-graining).\\
Force fields not only define the atomtypes and their properties, but also all parameters for the calculation of the potential, especially force constants, equilibrium distances a.o. Therefore they define the physics of the system.
\subsubsection{All-atom and the CHARMM36 force field}
CHARMM (\textit{Chemistry at HARvard Macromolecular Mechanics}) is a MD engine, which uses its own force fields. They are bunched together as CHARMM force fields and are provided for other MD engines as well. The number of the force field corresponds to the version number of CHARMM, in which the force field was used first. The force field CHARMM36 \autocites{charmm36_protein}{charmm36_lipids} (C36) was published in 2010.\\
\\
In C36 all atoms are considered (all-atom force field). Parameters were mainly optimized to structural experimental data, such as nuclear magnetic resonance (NMR) or X-ray data, but also QM and semi empirical QM calculations were used (e.g. for dihedral angles of the sidechains of proteins \autocite{charmm36_protein} and partial charges of lipids \autocite{charmm36_lipids}).\\
All simulations for optimizing have been done with a time step of 2$\si{\femto\second}$. Therefore very detailed dynamics are still included in the simulations.\\
\\
There are several models for water, which can be used with all-atom force fields. They differ mainly in intermolecular interactions of water atoms. For simulations in this thesis the TIPS3P water model was used. It is a modification of the TIP3P water model \autocite{tip3p} used in parametrisation of C36, which allows also Lennard-Jones-interactions on the hydrogens and not only on the oxygen \autocite{charmm36_protein}. % TODO: better citing?
\subsubsection{Coarse graining and the Martini 2.2 force field}
The Martini 2.2 \autocites{martini22_lipids}{martini22} force field was introduced in 2013 and is one of the most famous coarse graining force fields. It maps usually four heavy atoms onto a single bead (ring-like structures need a higher resolution), which implicates of course a loss of chemical information, but also an enormous reduction of computational cost. With this approach much larger time- and spatial scales are accessible for MD simulations.\\
\\
The parametrisation in Martini is mainly based on fitting partitioning free energies of small molecules, such as amino acid side chain analogues, between a range of polar and non-polar solvents to results from all-atom simulations and experimental. For lipids also thermodynamic properties, such as area per lipid, have been considered. The parameters for membrane-protein interactions were optimized by fitting binding energies of small peptides and a membrane to experimental data and results from all-atom simulations data. In the simulations for optimization a time step of 20$\si{\femto\second}$ up to 30$\si{\femto\second}$ was used \autocites{martini22}{martini22_lipids}.\\
\\
Coarse graining brings a lot of side effects a long. First of all, the coarse graining is not unique, therefore important structural properties (e.g. the lipid tail length) are neglected. In proteins the simplifications also lead to problems, because the secondary structure become less stable in coarse grained models and has to be constrained by elastic networks (additional bonds between backbone beads) \autocite[p. 6812]{martini22_check}.\\
Further more coarse grained beads have a larger size, which leads a.o. to a smoothing of the energy profile. Because smaller local minima in the energy profile, which would slow down the evolving of the system, are smoothed out, coarse graining speeds up the dynamics in the system. The speed up factor is not constant, but can be relatively good approximated by factor four (obtained in most diffusion simulations) \autocites[p. 6810]{martini22_check}[p. 7815]{martini}.\\
\\
Coarse graining also has an effect on the entropy and the temperature dependency of the system. In NpT ensembles the Gibbs free energy $G$ is given by
\begin{equation}
G = H - T S
\end{equation}
where $S$ is the entropy and $H$ the enthalpy. Due to a lower number of degrees of freedom in coarse grained systems the configurational entropy is reduced. Because Martini is tuned to free energy calculations, this implicates also a reduction of the enthalpy. Therefore a decomposition of $G$ might not be realistic \autocite[p. 6811]{martini22_check}.\\
\\
In Martini solvent beads represent four water molecules. Unfortunately the Martini water has a freezing temperature of up to 300 $\si{\kelvin}$ and the freezing process is very sensitive to nucleation. To prevent this antifreeze beads were introduced, which have the same properties as the standard water beads except for a larger $\sigma$ for the Lennard-Jones potential. By changing around 10\% of the standard solvents to these antifreeze beads, the lattice conformation is disturbed and the freezing temperature is increased \autocite[p. 7815]{martini}.\\
The treatment of water as uncharged beads also implies, that the water phase is not polarizable in Martini. This problem is addressed by assuming a uniform relative dielectric constant, but at water phase interfaces the electrostatic interactions are systematically wrong and the interaction strength of polar beads is underestimated. To overcome this a polarizable water (PW) model was introduced to Martini. PW consists of three beads with equal mass. There is one positively charged bead and one negatively charged, both are bound to the central neutral bead. The charged beads acting only electrostatical with other charged beads, but not intramolecular. To control the distribution of the dipole momentum the binding angle between the three beads is constrained. The central bead interacts via the Lennard-Jones potential \autocite{polarizableMartini}. However PW comes with a higher computational cost and is not as well tested as the standard water model.
\subsection{External constrains}
\subsubsection{Periodic boundary conditions}
\label{subsec:pbc}
Because the simulation of an open system is not possible, boundary conditions have to be considered. Closed boundaries often lead to surface interaction artefacts and are therefore in most cases not suitable for MD simulation. That is why usually periodic boundary conditions (PBC) are used.\\
To use PBC the shape of the simulation box has to have a space filling geometry (e.g. rectangular or rhombic dodecahedron). With periodic boundary conditions images of the simulation box are repeated in every direction. If a particle leaves the simulation box, its periodic image is coming in from the opposite. So the number of particles is kept constant while surface interactions are avoided.\\
A particle interacts only with the nearest image of another particle, which means that particles near boundaries can interact with periodic images of other particles instead of the real particle. Nevertheless molecules can have long range interaction with their own periodic images, which leads to artefacts and has to be considered when choosing the systems size.\\
It is generally thought, that PBC only have small or no effects on equilibrium properties or structures of fluids, but known problems are the absence of long wavelength fluctuations (e.g. in phase transitions) or the violation of angular momentum conversation \autocite[p. 141f]{greenBook}.
\subsubsection{Thermostats}
Integration schemes in MD are designed to conserve the energy of a system. This refers to a microcanonical ensemble, in which the number of particles $N$, the volume $V$ and the energy $E$ is constant (NVE). However in real biological systems not the energy but rather temperature is kept constant, which is the canonical ensemble (NVT). This can be achieved by thermostats. For the thermostats used in this thesis some characteristics are outlined below.
\paragraph{Berendsen thermostat}
The Berendsen thermostat \autocite{berendsen} couples the system weakly to a heat bath with temperature $T_0$ by scaling the velocities of the particles. The temperature $T$, to which the system is set to, is given by
\begin{equation}
\label{eq:berendsen_time_evolving}
\frac{\diff T}{\diff t} = \frac{T_0 - T}{\tau_T}
\end{equation}
Therefore an exponential relaxation can be observed. Weak coupling means, that the energy difference is not conducted in one rescaling, but over a given time scale $\tau_T$, which allows a specification of the coupling strength.\\
It has been shown, that rescaling of velocities transfer kinetic energy from internal degrees of freedom, such as vibrations, to translational and rotational kinetic energy of the center of mass of the system \autocite[p. 738]{velRescaleSucks}, which results in a wrong sampling of phase space. Therefore this thermostat is not suitable for production runs. Due to the exponential decay it adjusts quit fast, is stable against large deviations from the desired temperature and prevents oscillations. For these reasons it is often used in equilibration runs or non-equilibrium MD simulations \autocite[p. 3689]{berendsen}.
\paragraph{Parinello-Bussi thermostat}
The Parrinello-Bussi  thermostat \autocite{parinelloBussi} extends the Berendsen thermostat by a stochastic term, which leads to a more accurate sampling of phase space. The advantages of the Berendsen thermostat are still in place \autocite[p. 31]{gromacsManual}.
\paragraph{Nosé-Hoover thermostat}
The Nosé-Hoover thermostat \autocites{nosehooverthermo}{nosehooverthermo2} extends the Hamiltonian of the system by a friction representing a heat bath. The equations of motion are modified to
\begin{equation}
\frac{\diff^2 \vec{r}_i}{\diff t^2} = \frac{\vec{F}_i}{m_i} - \frac{p_\xi}{Q(T_0, \tau_T)}\frac{\diff \vec{r}_i}{\diff t}
\end{equation}
Here $\xi$ represents the friction parameter with momentum $p_\xi$ and a mass parameter $Q(T_0, \tau_T)$, which defines the coupling strength. $Q$ depends on the target temperature $T_0$ and a coupling time scale $\tau_T$. The evolution of $\xi$ is defined as
\begin{equation}
\frac{\diff p_\xi}{\diff t} = T - T_0
\end{equation}
This leads to oscillations between the system and the heat bath with the period $\tau_T$. The relaxation is about five times slower as by an exponential relaxation (see Berendsen or Parrinello–Bussi thermostats) with the same $\tau_T$ \autocite[p. 32f]{gromacsManual}.\\
\\
Often groups are coupled to independent thermostats. This is helpful, because the heat exchange between f.e. proteins and solvents is often not correct. Therefore proteins would cool down and the solvent would heat up \autocite[p. 34]{gromacsManual}.
\subsubsection{Barostats}
In biological systems often not the volume but the pressure is constant. This refers to the isobaric isothermal ensemble (NpT). Below the characteristics of the barostats used in this thesis are outlined.
\paragraph{Berendsen barostat}
Analogously to the Berendsen thermostat the Berendsen barostat \autocite{berendsen} couples the systems pressure $P$ weakly to an external pressure $P_0$ by rescaling the positions of the particles.\\
\begin{equation}
\label{eq:berendsen_pressure_time_evolving}
\frac{\diff P}{\diff t} = \frac{P_0 - T}{\tau_P}
\end{equation}
$\tau_P$ is the time scale of coupling. Similar to the Berendsen thermostat this barostat does not sample the NpT ensemble and should therefore only be used in equilibration runs and non-equilibrium MD simulations \autocite[p. 36]{gromacsManual}.
\paragraph{Parinello-Rahman barostat}
In the Parinello-Rahman barostat \autocites{parinelloBarostat}{parinelloBarostat2} the equations of motion of the particles change to
\begin{equation}
\frac{\diff^2 \vec{r}_i}{\diff t^2} = \frac{\vec{F}_i}{m_i} - \underline{M} \frac{\diff \vec{r}_i}{\diff t}
\end{equation}
where $\underline{M}$ is a matrix is given by a differential equation depending on the current pressure, the target pressure, the volume and a time scale. The Parinello-Rahman barostat samples the full phase space and can be therefore used in production runs. Large deviations from the desired pressure however lead to oscillations in the box, therefore it should not be used for equilibration runs \autocite[p. 36]{gromacsManual}.\\
\\
\gromacs\,provides the possibility to couple the z-direction independently from the x- and y-direction, which is called semiisotropic pressure coupling. This feature is useful for membrane and pulling simulations, because the dynamics differ a lot between these axes.
\section{Free energy calculations}
\section{Free energy}
To understand state transitions in a physical system the free energy is very handy quantity. It is directly linked to the probability distribution for different states and other quantities can be derived easily.\\
Below the basic concept of free energy is outlined as well as a practical method to retrieve free energy landscapes from MD simulation, which was used in this thesis.
\subsection{Partition functions and free energies}
The behaviour of a system depends on the interaction with the outside of the system. These interactions are classified in ensembles. The most important are the microcanonical ensemble ($N, V$ and $E$ constant), canonical ensemble ($N, V$ and $T$ constant) and the isothermal-isobaric ensemble ($N, p$ and $T$ constant).\\
\\
In the microcanonical ensemble the probability, that a system enters a microstate with energy $E' = E(\mathbf{q}, \mathbf{p})$ ($\mathbf{q}, \mathbf{p}$ are the positions and momenta of the particles respectively), is equal for all $\left|E' - E\right| < \diff E$ and $0$ else. Therefore the partition function $\Omega$ is given as
\begin{equation}
\Omega(N, V, E) = C_0 \int \delta\left(\mathcal{H}(\mathbf{q}, \mathbf{p}) - E\right) \diff \mathbf{q} \diff \mathbf{p}
\end{equation}
where $\mathcal{H}(\mathbf{q}, \mathbf{p}) = U(\mathbf{q}) + K(\mathbf{p})$ is the Hamiltonian and $C_0$ a proportional constant, in which the smallest phase space volume and the indistinguishability of particles have to be taken into account \autocite[16]{freeEnergyBook}.% For a set of $N$ identical particles this would be $\frac{1}{h^{3N} N!}$ with $h$ the Planck constant.\\
\\
\\
In the canonical ensemble however the temperature $T$ is kept constant instead of the energy. Therefore the partition function has to include all possible energies weighted with their probability given by the Boltzmann factor. Below $\frac{1}{k_B T}$ is shortened with $\beta$.
\begin{align}
Q(N, V, T) &= \int \exp\left(-\beta E\right) \Omega(N, V, E) \diff E\\
&= C_0 \int \exp\left(-{\beta\mathcal{H}(\mathbf{q}, \mathbf{p})}\right) \diff \mathbf{q} \diff \mathbf{p}
\end{align}
The configurational integral is defined as 
\begin{equation}
Z(N, V, T) = \int \exp\left(-{\beta U}(\mathbf{q})\right) \diff \mathbf{q}
\end{equation}
It is important to see, that $\mathcal{H}$ only depends on the quadrature of $\mathbf{p}$, so the integral over the momenta can always be solved analytical by turning it into a Gauss integral. This implies, that for two related systems, in which the particle masses are the same, the integral over $\mathbf{p}$ does not change and therefore
\begin{equation}
\frac{Q_2}{Q_1} = \frac{Z_2}{Z_1}
\end{equation}
holds \autocite[17]{freeEnergyBook}.\\
\\
The partition function of the isothermal-isobaric ensemble, in which the pressure is kept constant instead of the volume, can be set up by expanding the Hamiltonian by the work the system is doing on expansion/contraction.
\begin{align}
\mathcal{H}' &= \mathcal{H} + pV\\
\Xi(N, p, T) &= C_1 \int \exp\left(-\beta p V\right) Q(N, V, T) dV
\end{align}
where $C_1$ is a volume scale. For a further discussion of $C_1$ please \autocite[see][]{isothermalIsobaric}.\\
\\
The Helmholtz free energy $A$ refers to a canonical ensemble, which means that it is minimal in equilibrium under constant temperature and volume. The Gibbs free energy $G$ refers to an isothermal-isobaric ensemble \autocite[19]{freeEnergyBook}.
\begin{align}
A &= - \beta \ln\left(Q\right)\\
G &= - \beta \ln\left(\Xi\right)
\end{align}
Usually the exact value of the free energy is unknown, but the main interest lies in free energy differences between two states of a system. For the Helmholtz free energy $A$ this difference is given as
\begin{equation}
\Delta A = A_2 - A_1 = - \beta \ln\left(Q_2\right) + \beta \ln\left(Q_1\right) = -\beta \ln\left(\frac{Q2}{Q1}\right) = -\beta \ln \left(\frac{Z_2}{Z_1}\right)
\end{equation}
\subsubsection{Free energy in MD and Umbrella Sampling}
The free energy of a system as a function of a set of parameters $\vec{\xi}$ is given as
\begin{equation}
\label{eq:free_energy_from_rho}
A(\vec{\xi}) = - \beta \ln\left(\rho(\vec{\xi})\right)
\end{equation}
$\rho(\vec{\xi})$ can be easily measured during a MD simulation by counting the number of states, in which $\vec{\xi}(\mathbf{q}) = \vec{\xi}$. Therefore it is also referred as histogram. $\vec{\xi}$ are often called reaction coordinates and could be e.g. the distance between two molecules.
In reality however $A(\vec{\xi})$ can have big barriers and because the potential energy $U$ is sharply distributed around its mean in $NVT$ simulations, $\vec{\xi}$ can only hardly be sampled during a finite simulation.\\
\\
One possibility to overcome the sampling problem is umbrella sampling \autocite{originalUmbrellaSampling}. In this approach the path $\xi$ (for simplicity one dimensional) is split up into distinct windows $[\xi_0, \xi_n]$. To each window $i$ a biasing potential $\hat{U}_i(\xi)$ can be applied to ensure a good sampling of $\xi$ around $\xi_i$. This changes the potential energy to
\begin{equation}
U_{B, i}(\mathbf{q}) = U(\mathbf{q}) + \hat{U}_i(\xi(\mathbf{q})))
\end{equation}
After a simulation with the biased potential $U_{B, i}$ the unbiased probability distribution $\rho_i(\xi)$ has to be reconstructed from the observed biased one $\rho_{B, i}(\xi)$.\\
In the following explanation the window index $i$ is dropped and the unbiased system and the biased system are referred as 1 and 2 respectively.\\
The probability density of finding the biased system in a configuration, in which its potential energy differs from the potential energy of the unbiased system (that is in the same configuration), by $\Delta U = U_2(\mathbf{q}) - U_1(\mathbf{q})$ shall be considered. This implies $U_2(\mathbf{q}) = U_1(\mathbf{q}) + \Delta U$, which is done by an Delta function.
\begin{align}
\rho_2(\Delta U) = \frac{\int \exp\left(-\beta U_2(\mathbf{q})\right) \delta\left(U_2(\mathbf{q}) - U_1(\mathbf{q}) - \Delta U\right) \diff \mathbf{q}}{Z_2}
\end{align}
By substituting $U_2(\mathbf{q})$ the factor $\exp\left(-\beta\Delta U\right)$ can be moved out of the integral. After a multiplication of $\frac{Z_1}{Z_1}$, $\rho_1(\Delta U)$ can be identified. $\rho_1(\Delta U)$ has the same meaning as $\rho_2(\Delta U)$, but refers to the unbiased system. This is the corrected probability density \autocite[p. 179ff]{UnderstandingMD}.
\begin{align}
\label{eq:rel_rho2_rho1}
\rho_2(\Delta U) = \frac{Z_1}{Z_2} \exp\left(-\beta\Delta U\right) \rho_1
\end{align}
Because $\Delta A = -\beta \ln\left(\frac{Z_2}{Z_1}\right)$, \autoref{eq:rel_rho2_rho1} can be transformed into
\begin{equation}
\rho_1(\Delta U) = \exp\left(-\beta\left(\Delta A - \Delta U\right)\right)\rho_2(\Delta U)
\end{equation}
Therefore the measured biased histograms $\tilde{\rho}_{B, i}(\xi)$ can be turned into unbiased histograms $\tilde{\rho}_{i}(\xi)$ via
\begin{equation}
\label{eq:reconstruction_from_biased}
\tilde{\rho}_{i}(\xi) = \exp\left(-\beta\left(\Delta A_i - \hat{U}_i(\xi)\right)\right)\tilde{\rho}_{B, i}(\xi)
\end{equation}
Of course $\Delta A_i$ is not known, but assuming that $\rho(\xi)$ is a continuous function, the results from the single windows can be combined and afterwards normalized. With this method however one can only have two histograms in the overlapping region. Another problem is, that the sampling in the tails is usually poor and statistical errors, which propagate through all overlapping regions, can become very large \autocite[236ff]{freeEnergyBook}. Therefore umbrella sampling is usually combined with the Weighted histogram analysis method (WHAM) \autocites{originalWHAM, extensionWHAM}. This algorithm is able to combine several histograms in one overlapping region and it is designed to keep the statistical errors small. The main idea is to combine probability densities linearly with an additional weighting factor $\omega_i(\xi)$ to the total probability density $\tilde{\rho}$. The weighting factors $\omega_i$ are chosen iteratively in a way, that the overall statistical error is minimized.\\ 
\section{Contact Analysis}
\subsection{Intromolecular}
The interface of the FERM domain and the kinase is of special interest since the domains have to dissociate for a full activation. Therefore contacts and contact areas at the interface are investigated using the following methods.\\
\subsubsection{ConAn}
ConAn is a completely, bug free, user friendly and well documented tool to analyze the distances between all residue pairs in a molecule. It is great, believe me, and totally works out of the box. \textit{Some explanation about Contact maps? But should be self explaining :D}
\subsubsection{Contact area}
The contact area of the interface is determined with the solvent accessible surface area (SASA) and is defined as
\begin{equation}
	\text{CA} = \frac{1}{2} \left(\text{SASA}_\text{FERM} + \text{SASA}_\text{kinase} - \text{SASA}_\text{FERM-kinase}\right)
\end{equation}
The calculation of the SASA values was done with \gromacs{} sasa tool.
\subsection{Intermolecular}
In order to determine the cluster behaviour of FAK intermolecular contacts of FAK molecules have to be considered. For this purpose, the following terms are defined.
\paragraph{Interaction} Proteins or parts of proteins interact, if their minimal distance is smaller than a cut-off distance (here $1.5\,\si{\nano\metre}$).
\paragraph{Neighbour} Protein A is a neighbour of Protein B, if they are interacting. One protein can have several neighbours. For a more detailed characterisation the following two-molecule interaction types are defined (see also \autoref{methods:inttypes}):
\begin{enumerate}[label={type \theenumi:}, leftmargin=*]
	\item only the FERM domain interacts with only the FERM domain of the other protein
	\item only the kinase interacts with only the kinase of the other protein
	\item only the FERM domain interacts with only the kinase of the other protein
	\item the FERM domain is interacting with both, the FERM and kinase of the other protein
	\item the kinase is interacting with both, the FERM and kinase of the other protein
	\item the FERM domain is interacting with the FERM domain of the other protein and the kinase is interacting with the kinase of the other protein
	\item the FERM domain is interacting with the kinase of the other protein and the kinase is interacting with the FERM domain of the other protein
\end{enumerate}
\paragraph{Cluster} Protein A belongs to a cluster, if it has at least one neighbour inside the cluster. Two neighbouring proteins form a cluster of size 2. One protein can only belong to one cluster.
%
%
%
\begin{figure}
	\centering
	\includegraphics[width=\textwidth]{figures/introduction/classification}
	\nicecaption{Different two-molecule interaction types}{The black part refers to the FERM domain, the white to the kinase.}
	\label{methods:inttypes}
\end{figure}
%
%
%
\chapter{Setup}
\section{Protein structure}
%The present thesis is meant to give an insight into dynamics and characteristics of FAK-FAK interactions as well as interactions with a \pip{} containing membrane obtained from MD simulations. For this purpose the Martini force field and C36 were used.\\
All simulations were made with starting configurations adapted from previous work in the group (C36 forcefield: \textcite{pap003}, Martini forcefield: \textcite{SARA}). These configurations contain only a FERM-kinase fragment without the FAT domain (residues 35 to 686, PDB 2J0J \autocite{structFAK}).\\
As explained in \autoref{subsub:coarsegraining} the secondary structure of proteins have to be stabilized in Martini using elastic networks. This was set up by \textcite{SARA}. It acts only between residues of the same domain with a force constant of 830 $\si{\kilo\joule\mole^{-1}\nano\meter^{-2}}$. Therefore the interface between FERM domain and kinase is not affected and the linker is still flexible.
\section{Free energy of basic patch}
In the first part of this thesis the free energy landscape of the \pip{} binding to the basic patch was retrieved for C36, Martini and Martini with PW using umbrella sampling.\\
For simplicity only a part of the F2 subdomain (residues 107 to 219, from here on F2 lobe), which contains the basic patch, was used. The lobe was placed above a \pip{} embedded into a phosphatidylethanolamine (\pope{}) membrane (see \autoref{tobeadded}). With a provided transformation tool \autocite{backwardpy} the whole system was transferred to a Martini structure , to which the elastic network was applied.\\
After a short equilibration the protein was pulled slowly away from the membrane using a distance pull between the COM of the protein and the COM of the \pip{}. This simulation was used to retrieve starting conformations for the umbrella windows (between 90 and 120, dependent on the resulting sampling). Afterwards each umbrella window was equilibrated for 0.5$\si{\nano\second}$ and then simulated for 6$\si{\nano\second}$ (10$\si{\nano\second}$ for Martini and Martini with PW). For each forcefield the pulling and umbrella sampling was done five times to estimate the statistical error.\\
The calculation of the free energy was done with \gromacs{} implementation of WHAM \autocite{gromacsWHAMpaper}.
The presented results are based on a total simulation time of $6.33\,\si{\micro\second}$ for Martini, $5.64\,\si{\micro\second}$ for Martini with PW and $3.88\,\si{\micro\second}$ for C36. % C36: 1=180, 4.2=135, 5=102, 5.2=102, 6=127, Mart: 1=134, 2=126, 3=130, 4=113, 5=130, MartPol: alle=113
\section{FAK clustering on membrane}
\label{setup:fakcluster}
The second part of this thesis addresses the interactions between different FAK. Therefore 25 proteins were simulated on a membrane.\\
The starting configurations for this system was built of single frames from simulations of a single FAK on a phosphatidylcholine (\popc{}) and \pip{} (15\%) membrane. The frames were chosen equidistant in time over a time span of $50\,\si{\micro\second}$.\\
\\
\textcite{SARA} has done simulations with the same setup in previous work, in which she observed an unexpected behaviour of the protein inclination. This inclination shall be quantified as the angle $\beta$ between the z-axis and the vector $\vec{d}_F$ connecting the F1 and F2 subdomain.
\begin{equation}
	\cos\left(\beta\right) = \frac{\vec{d}_{F, z}}{d_F}\quad d_F = \left|\vec{d}_F\right|
\end{equation}
The distribution of $\beta$ for each of the five different copies can be seen in \autoref{TOBEADDED}. Each copy was simulated for $10\,\si{\micro\second}$.\\
The red line is from a trajectory, in which the protein fell on the membrane (resulting in a mean value of $90\,\si{\deg}$). There are several reasons why this is rather an artefact of the Martini force field than a possible binding pose of FAK to the membrane as suggested by \textcite{pap002}. The first one is, that in our simulation the interacting residues proposed by \textcite{pap002} are on top of the FAK and not in contact with the membrane (see \autoref{Something cool}). Indeed contact analysis showed, that virtually all residues (in FERM and kinase) were interacting with the membrane. Another one is, that this behaviour was not observed in equivalent all atom simulations in C36 ($1.5\,\si{\micro\second}$ in total). Here two maxima were observed around $8\,\si{\deg}$ and around $20\,\si{\deg}$ (similar to the blue and green curve in \autoref{tobeadded}), the largest observed angle was $40\,\si{\deg}$.\\
Therefore the inclination was restricted in the simulations for this thesis. This was achieved by applying an external force to the FERM domain. This force is acting onto the F1 and F2 subdomains parallel to the z-axis. It is proportional to the deviation of their z-distance $\Delta z$ from a reference distance $z_0$. An illustration of the force can be found in \autoref{tobeadded}.\\
For the determination of $z_0$ only the green and the blue distribution were considered, because the large angles observed in the other distributions have not been observed in C36 simulations. The mean value of $\vec{d}_{F, z}$ for these two distributions is $2.228\,\si{\nano\metre}$, which was therefore set as $z_0$.\\
% TODO: ask Csaba, why we chose 100 as force constant.
\\
The frames were arranged (see \autoref{tobeadded}) and the resulting system shortly equilibrated. Afterwards it was simulated for $10\,\si{\micro\second}$. Five copies of the system with different arrangements of the starting frames were used resulting in a total simulation time of  $50\,\si{\micro\second}$. The trajectories are called During the whole simulation each protein was stabilized with the force described above.\\
\\
For the simulation options the recommended values were chosen for all simulations and can be found IN THE APPENDIX. All simulations have been done for a temperature of $300\,\si{\kelvin}$.
\chapter{Results}
\section{FAK in solution}
% TODO: link to methods
In this section the conformation of FAK in absence of \pip{} and other FAK molecules is analyzed. For this purpose the simulation data of setup 1 were used (see \autoref{setup:setup1}).\\
\\
First the COM distances of F1 to the N-lobe ($d_\text{F1-N}$) and F2 to the C-lobe ($d_\text{F2-C}$) are considered. In \autoref{tobeadded} a hexagonal binning plot of both values can be found, which indicates, that there are two different states: one in which $d_\text{F2-C}$ is larger (spot 1) and $d_\text{F1-N}$ smaller and one the other way around (spot 2). The systems starts in spot 1 and goes to spot 2 after $7\,\si{\micro\second}$ (this transition goes along with several frequent transitions), where it stays until the end of the simulation.\\
\textcite{jing} performed equivalent simulations in C36 and obtained
\begin{align}
	d_\text{F1-N} &= (3.30 \pm 0.40\,\text{std})\si{\nano\metre}\\
	d_\text{F2-C} &= (3.15 \pm 0.15\,\text{std})\si{\nano\metre}
\end{align}
These values can not be classified in one of the two spots as both distances are lower than those obtained in Martini.\\
\\
A contact map of the interface between the FERM domain and the kinase for frames of spot 2 can be found in \autoref{tobeadd}. Two contact areas can be identified. The first one (area 1) is located between F1 and the N-lobe/activation loop. It shows i.e. contacts between \acid{Y}{576} and \acid{Y}{577} and residues of the FERM domain. The minimal distance in this area, between residue \acid{H}{41} and \acid{Y}{576}, is $0.45\,\si{\nano\metre}$ with an RMSF value of $0.03\,\si{\nano\metre}$. This reflects the burying of the activity regulating residues in closed state.\\
The second contact area (area 2) is located between F2 and the C-lobe. The spots occur around the residues \acid{Y}{180} and \acid{D}{200} of F2 as well as \acid{F}{596} and \acid{R}{665}. The minimal distance in this area occurs between \acid{Y}{180} and \acid{F}{596} with $0.45\,\si{\nano\metre}$ and an RMSF value of $0.02\,\si{\nano\metre}$. These two residues have been reported as important actors in the interface by showing, that a mutation disturbs the interface and enhances the activation of FAK.\\
The linker show contacts with both domains. Interestingly the minimal distance in the marked areas (area 3 and area 4) occur between the autophosphorylation site \acid{Y}{397} and \acid{H}{58} ($0.45\,\si{\nano\metre}$, RMSF $0.03\,\si{\nano\metre}$) in F1 or \acid{Y}{576} ($0.50\,\si{\nano\metre}$, RMSF $0.10\,\si{\nano\metre}$) in the kinase. This is consistent with the concept, that autophosphorylation is prevented in closed conformation by a binding of the linker to the FERM domain.\\
\\
In contrast to \autoref{tobeadd}, the contact map for frames of spot 1 show less contacts between F2 and the C-lobe, i.e. around the mentioned residues \acid{Y}{180} to \acid{M}{183}. A few additional contacts appear between F1 and the N-lobe, but these are only minor spots.
\section{FAK binding to \pip{}}
In this section the impacts on the conformation due to a binding to a \pip{} containing membrane are investigated and compared to results from \autoref{sec:fak_sol}.\\
First of all measurements of the free energy profile of the basic patch bound to a \pip{} are presented. Afterwards simulations of setup 4 are considered. Because a stabilizing force was applied to the FAK molecules in these simulations, the impacts of the force are estimated before conformational changes are investigated.
From the umbrella simulations the free energy profile was retrieved using \gromacs{} WHAM. In \autoref{bild} the average of the five copies together with the standard deviation can be found for each forcefield. The range between 6$\si{\nano\metre}$ and 7$\si{\nano\metre}$ was set to 0.\\
As you can see in the plot, the results for C36 and Martini are very similar: both show a depth of $\approx 17.5 \si{\kilo\cal/\mole}$ and the same slope between 3$\si{\nano\metre}$ and 4$\si{\nano\metre}$. However it seems like the Martini profile is shifted to smaller z-distances. One reason for this is, that due to the coarse graining the COM distances alters (in our case the COM distance of Protein and \pip{} decreased by 0.1$\si{\nano\metre}$ after coarse graining). Beyond 4$\si{\nano\metre}$ there is a larger difference: while the free energy profile flattens out in C36, a kink can be found in Martini. This is explained by the different treatment of long range electrostatic interactions. For this Martini just uses a cut off radius, C36 uses PME.\\
Also Martini with PW uses PME for long range electrostatic interactions and indeed this fits much better to the profile observed in C36. However the binding strength is drastically underestimated when using PW. The same effect was observed by simulations of a single FAK on a \pip{} containing membrane. Here the kinase dissociated from the membrane, which is probably due to an underestimation of the \pip{} binding site in the kinase.\\
The results indicate, that Martini is a suitable simplification for simulating FAK on membranes, which allow much longer simulations of much larger systems.
\label{forceana}
In setup 3 and setup 4, we stabilised each protein with an external force acting on the FERM domain (see \autoref{motivation}). To ensure that this restraining does not cause major artefacts, the force-dependency of the used observables is examined below.\\
\\
The force has a mean value of $2.44\,\forceunit$ and a standard deviation of $21.80\,\forceunit$. Interestingly, the positive mean value shows that the proteins are rather pushed downwards than pulled up. This indicates that the stabilising force does not have to hold the protein upright the whole time. % TODO: clarify?
\\
Linear regressions show that all of the quantities $d_F$, $d_\text{F1-N}$, $d_\text{F2-C}$ and CA have a negligible correlation to the applied force. Here, negligible means that either the regression result was not significant or that the obtained slope was so small, that a change of one standard deviation in force would not change the quantity noticeably.\\
\\
We also tested the inter-residue distances for correlation with the applied force. To this end, 10 different proteins without neighbours were monitored, each for $1\,\si{\micro\second}$. For each residue pair in this dataset, we performed a linear regression. \autoref{force:contactmap} shows the calculated Pearson correlation coefficient (only significant correlations with Pearson $\left|r\right| > 0.3$). The mean value of the slope for the positively correlated pair distances is $33.7\,\si{\pico\newton/\nano\metre}$ and $-32.7\,\si{\pico\newton/\nano\metre}$ for the negatively correlated pair distances. Thus, the force can influence residue pairs contributing to the interface. However, we do not expect major changes in the contact regions, since the majority of residue pairs show only weak correlations.\\
\\
In summary, we do not expect large perturbations of our observables due to the force. However, this approach has limitations. First, there could be binding poses in multiple FAK interactions requiring a large inclination of the FAK molecule. These states would be suppressed by the force. Another limitation is that the force does not only prevent tilts around the long axis of FAK (falling to sideways), but also around the short axis which happens e.g. in FERM-FERM dimers. Lastly the reference to the z-axis is problematic. A reference to the membrane might be better, because it would involve membrane curvature as well.
%
%
%
\begin{figure}
	\centering
	\includegraphics[width=.7\textwidth]{figures/results/interface_corr}
	\nicecaption{Correlations in contact map}{The contact map shows correlations between inter-residue distances and the stabilising force. The mean slope is $33.2\,\si{\pico\newton/\nano\metre}$, the maximal Pearson $r < 0.5$.}
	\label{force:contactmap}
\end{figure}
%
%
%

\subsection{Conformational changes}
In this section the conformation of FAK bound to \pip{} (FAK-PIP) is compared to the observations from \autoref{sec:fak_sol} (FAK-SOL). For this purpose the simulation data of setup 4 was used with the condition, that other FAK molecules are more than $2\,\si{\nano\metre}$ away (0 neighbours). The contact map is based on the same dataset, which was used in \autoref{forceana:intramolec}.\\
\\
Analogously to \autoref{sec:fak_sol} the distribution of the COM distances is presented in \autoref{tobeadd} as an hexagonal binning plot. Again, different spots can be obtained. The spot with most encounters (spot 1) as well as the second spot (spot 2) are located at small values for both, $d_\text{F1-N}$ and $d_\text{F2-C}$. These spots show also smaller distances than FAK-SOL. In addition to this two spots appear, one at larger $d_\text{F2-C}$ (spot 3) and one at larger $d_\text{F1-N}$ (spot 4). These are however less populated and not as concentrated as spot 1 and 2. While spot 4 could be identified with spot 2 of FAK-SOL, spot 3 show completely new values for both COM distances.\\
\\
In \autoref{tobeadded} the difference of the contact maps of FAK-SOL and FAK-PIP can be found, again only for the interface. The distances between F2 and the C-lobe tend to get smaller, even if contacts around \acid{R}{665} show another trend. Also the contact between F1 and the N-lobe/activation loop show smaller distances. The RMSF values are increasing for all residue pairs.\\
Remarkable changes occur in the linker region. The residues around the autophosphorylation site \acid{Y}{397} increase their distances to residues \acid{M}{384} to \acid{S}{390} by up to $0.9\,\si{\nano\metre}$. Also the RMSF values are increased in the linker by up to $0.40\,\si{\nano\metre}$ near \acid{Y}{397}.\\
\\
With these observations the enhanced autophosphorylation of FAK bound to \pip{} can be comprehended. However the impacts on the interface between the FERM domain and the kinase are quit small. One reason for this could be the applied elastic network. Since electrostatics are treated with a cutoff radius, long range conformational changes have to be transferred along the residues. Therefore the choice of the correct force constant is of large importance to obtain allosteric effects in Martini.
\section{Multiple FAK interactions}
In this section the interactions occuring between multiple FAK molecules are analysed, for which the data from setup 4 is used. At this point the reader shall be reminded, that the used protein structure lacks the FAT domain, which is in full length FAK connected to the kinase via a linker region. This might make an important difference for clustering processes.
\subsection{Structure of FAK oligomers}
\label{mult:oligs}
The characterisation of the emerged FAK clusters is very difficult as they differ a lot in size and shape. The largest cluster observed in setup 4 had a size of 21 proteins, while there are other proteins, which did not join any cluster at all. Present shapes of the clusters include long chains as well as ring like conformations or just an agglomeration (see \autoref{tobeadded}). \\
\\
First of all the mean number of neighbours is examined. One can see in \autoref{mult:nng_vs_t} a fast rising in the number of neighbours in the beginning and a flattening after $6\,\si{\nano\second}$. The average of the five copies is at the end of the simulation $1.86$. This indicates a tendency to chains of FAK.\\
In \autoref{mult:inttype_vs_t} the average number of encounters of the different interaction types is plotted against the time. It shows, that FERM-kinase interactions (type 3) occur the most, while the others occur equally often. Only type 6 and type 7 (interactions, in which all four domains are involved) occur much less than the others.\\
From these observations one could draw the conclusion, that the preferred arrangement of FAK molecules is a chain, in which the FERM domain interacts with the kinase of the next molecule (FK-Chain). A second possibility would be a chain, in which the FERM domain interacts with the FERM domain of the next molecule, while the kinase interacts with the kinase of the previous molecule (FFKK-Chain). Assuming a FAK chain of length $n$, FK-Chains would show $n$ encounters of type 3 interactions, FFKK-Chains only $n/2$, but for both type 1 and type 2. This would also be consistent with the observed data.
%
%
%
\begin{figure}
	\subcaptionbox{Mean number of neighbours against the time\label{mult:nng_vs_t}. Filled area is observed minimum and maximum.}[0.49\textwidth]{
		\includegraphics[height=5cm]{figures/results/averagenumber}
	}\hfill%
	\subcaptionbox{Mean number of interactions against the time\label{mult:inttype_vs_t}. Filled area is observed minimum and maximum.}[0.49\textwidth]{
		\includegraphics[height=5cm]{figures/results/multiple_typevstime}
	}%
	\phantomsubcaption
\end{figure}
%
%
%
\subsection{Activation due to clustering}
At last the impacts of clustering on FAK activation are addressed. Activation means here the dissociation of the FERM domain from the kinase, therefore the obtained trajectories are analysed with respect to the contact area (CA) of the FERM-kinase interface as a key quantity. Unfortunately in no FAK molecule a full dissociation took place at any time, therefore only trends can be considered at this point.\\
%
%
%
\begin{figure}
	\centering
	\includegraphics[height=6cm]{figures/results/nng_ca}
	\captionof{figure}{CA for different number of neighbours}
	\label{mult:nng_ca}
\end{figure}
%
%
%
At first glance the CA seems to be independent of the number of neighbours a protein has as well of the interactions it is in (see explanatory for the number of neighbours \autoref{mult:nng_ca}). Therefore the data has to be filtered more.\\
Motivated from \autoref{mult:oligs}, only FAK molecules inside chains are taken into account. A FAK molecule can be seen as a chain member, if it has exactly two neighbours and if these neighbours are not neighbours of one another. For FK-Chain only type 3 interactions were allowed, for FFKK-Chains both, type 1 and type 2. The resulting distribution of CA as well as the COM distances $d_\text{F1-N}$ and $d_\text{F2-C}$ can be found in \autoref{mult:fk_ca} for FK-Chain and in \autoref{mult:ff_ca} for FFKK-Chain.\\
As one can see in the plots FK-Chains do not have an influence upon the CA. Also the distribution of $d_\text{F1-N}$ and $d_\text{F2-C}$ is very similar to the one obtained in \autoref{mem:comdist}, except for less sampling. However in FFKK-Chains the mean CA value is $2\,\si{\nano\metre}$ smaller than the one for FAK-MEM. Also the $d_\text{F2-C}$ seems to be populated more at larger values. Nevertheless all these changes are very small.
%
%
%
\begin{figure}
	\centering
	\includegraphics[height=6cm]{figures/results/fk_ca}
	\captionof{figure}{Analysis of the FERM-kinase interface in FK-Chains. The blue line was obtained from FAK-SOL.}
	\label{mult:fk_ca}
\end{figure}
%
%
%
%
%
%
%
\begin{figure}
	\centering
	\includegraphics[height=6cm]{figures/results/ff_ca}
	\captionof{figure}{Analysis of the FERM-kinase interface in FFKK-Chains. The blue line was obtained from FAK-SOL.}
	\label{mult:ff_ca}
\end{figure}
%
%
%
%


\end{document}